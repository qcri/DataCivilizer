% !TEX root = ../main.tex
\section{Tractable Curation Workflow}
\label{sec:workflow}

The process of data curation may entail multiple iterations of discovery,
stitching, querying, and curation. Cleaning and
transformation procedures must be guided by the user. \dcv will use a fairly
conventional workflow system that will allow a human to construct sequences of
operations, undo ones that are unproductive, and utilize branching to try
multiple processing operations.


In addition, we plan to build a workflow orchestrator which will retain previous
operation sequences and propose ones that best fit the current situation. We
expect this tactic to be successful because the types of data curation and
preparation steps in an enterprise often follow repetitive patterns.
Specifically, our workflow orchestrator will store the user query, the sequence
of operations in a central workflow registry together with the meta-data and
signatures provided by the data discovery component. 


The orchestrator will evaluate the current user query and the initial results of
the data discovery component against the workflow registry. Similar queries and
similar data profiles are likely to demand similar cleaning procedures. Previous
workflows will be ranked based on the similarity of the user query and retrieved
discovery results. Accordingly, the related curation and preparation procedures
will be shown in ranked order to the user.
