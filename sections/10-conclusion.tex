% !TEX root = ../main.tex
\section{Conclusion and Future Work}
\label{sec:conclusion}

In this paper, we propose an end-to-end big data management system named \dcv
with the goal of discovering relevant data to the user specific tasks, link the
relevant data for better usage, cleaning the data with limit budget, handle data
updates, and enable iteratively processing the data. We build a module for each
of the requirement above. As data in large companies are usually across multiple
data storage platforms, we build our system in a polystore architecture.
Moreover, we propose to place the data cleaning operation in the query plan, and
trade-off the query result quality with the cleaning costs. A workflow orchestrator will leverage knowledge about previous workloads to propose relevant opertions to the user.
We deployed our preliminary system on two different institutions, MIT and Merck and got positive
feedbacks from the users.

% \dcv is a new end-to-end system that integrates the tasks of
% discovering relevant data, stitching it together in meaningful views for the
% users and integrating cleaning and curation along the process. We observe
% several challenges such as taming the scale on discovering data, finding
% efficient methods to stitch potentially dirty data and clean data under a budget
% with the goal of maximizing value for the user. We have built and deployed an
% initial prototype in two large organizations and we will demo the system at the
% conference. Many opportunities for future work lie ahead:

% \begin{myitemize}
% \item\textbf{Workflow management.} \dcv supports multiple
% iterations of discovery, view creation and data curation, all intertwined with
% manual human intervention. Currently we have a rudimentary workflow orchestrator
% that we plan to extend with further options such as: 
% i)~a recommendation mechanism to propose operations based on a workflow-aware log of past sequences;
% and ii)~inspect past operations, delete and highlight the promising ones through a
% lineage mechanism, facilitating other users' operations. 
% \item\textbf{Error model.} We plan to include a better error model based 
% on some of our recent findings in~\cite{datacleaning}. 
% The general idea is in capturing the \emph{importance} of
% errors, in particular, as referred to their impact on the final result quality.
% \item\textbf{Steering cleaning engine.} Cleaning happens across many layer. 
% We are currently testing different algorithms that we plan to
% integrate in the platform with the aim of helping users to
% clean the data that yields the higher quality returns.
% \end{myitemize}

