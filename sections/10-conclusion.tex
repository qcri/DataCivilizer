% !TEX root = ../main.tex
\section{Conclusions}
\label{sec:conclusion}

In this paper, we presented \dcv,  an end-to-end big data management system. 
\dcv aims to support  the discovery of data that is relevant to specific user tasks, 
the linkage of the relevant data for allowing complex polystore queries, 
the cleaning of the data under limited budget, 
the handling of  data updates, 
and the iterative processing of the data. 
%We built a module for each of the requirement above. 
As data in large companies are usually scattered across multiple
data storage platforms, we proposed the use of a polystore architecture to deploy our system.
A key characteristic of \dcv is that it places the data cleaning operations in the query plan
while  trading-off the query result quality with the cleaning costs. 
%The \dcv workflow orchestrator leverages knowledge about previous workloads to propose relevant operations to the user.
We have deployed our preliminary system at two different institutions, MIT and Merck and obtained positive
feedback from the users.

\section{Acknowledgments}

 Michael Stonebraker contributed significantly to the work presented in this paper;
 he precluded from being an author because of CIDR rules concerning PC chairs.

% \dcv is a new end-to-end system that integrates the tasks of
% discovering relevant data, stitching it together in meaningful views for the
% users and integrating cleaning and curation along the process. We observe
% several challenges such as taming the scale on discovering data, finding
% efficient methods to stitch potentially dirty data and clean data under a budget
% with the goal of maximizing value for the user. We have built and deployed an
% initial prototype in two large organizations and we will demo the system at the
% conference. Many opportunities for future work lie ahead:

% \begin{myitemize}
% \item\textbf{Workflow management.} \dcv supports multiple
% iterations of discovery, view creation and data curation, all intertwined with
% manual human intervention. Currently we have a rudimentary workflow orchestrator
% that we plan to extend with further options such as: 
% i)~a recommendation mechanism to propose operations based on a workflow-aware log of past sequences;
% and ii)~inspect past operations, delete and highlight the promising ones through a
% lineage mechanism, facilitating other users' operations. 
% \item\textbf{Error model.} We plan to include a better error model based 
% on some of our recent findings in~\cite{datacleaning}. 
% The general idea is in capturing the \emph{importance} of
% errors, in particular, as referred to their impact on the final result quality.
% \item\textbf{Steering cleaning engine.} Cleaning happens across many layer. 
% We are currently testing different algorithms that we plan to
% integrate in the platform with the aim of helping users to
% clean the data that yields the higher quality returns.
% \end{myitemize}

