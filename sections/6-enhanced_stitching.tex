% !TEX root = ../main.tex


\section{Querying the Join Graph}\label{sec:enhancedstitching}


After we ran the data stitching and get the join graph, the user can query the join graph explicitly or implicitly. The user can specify several attributes he is interested in, which compose a \emph{query schema}, and several predicates on these attributes. The query result is a view that subsumes the query schema and obeys the predicates. To answer a query, data civilizer generates a steiner tree for the query schema from the join graph. As there are multiple steiner trees, to select the best one, we give a cleanliness model to estimate the quality of the corresponding query result in Section~\ref{subsec:model:recall}. Given a steiner tree, we discuss how to maximize the quality of the corresponding result with a given budget in Section~\ref{subsec:gain}.


%Consider a \emph{path query} to be a predicate $P$ on an attribute in one of the tables along with a chain of PK/FK relationships, terminating with a retrieval of a non-key attribute in the final table. 



\subsection{The Cleanliness Model}\label{subsec:model:recall}

The result of a query is a view by joining all the tables in the steiner tree and filtering with the given predicates. Assume $\R$ is the desired table, i.e. the ground truth, and $\R'$ be the one generated from our join graph. Let  $|\R|$ be the number records in $\R$ and $|\R\cap\R'|$ be the number common records shared by $\R$ and $\R'$. We respectively define the precision and the recall of the query result as
$$Precision=\frac{|\R\cap\R'|}{|\R|}~and~Recall=\frac{|\R\cap\R'|}{|\R'|}.$$
Next we give a cleanliness model to estimate the two values.



Consider two tables $A$ and $B$ with FK-PK relationship in the steiner tree. Let $A.a$ be the foreign key and $B.b$ be the primary key. Data stitching will give us a maximum bipartite matching between the values in $A.a$ and $B.b$ where the weight of an edge is the likelihood the two end points represent the same entity. We can join all the tables in the steiner tree by the maximum bipartite matching to achieve $\R$ and $|\R|$. Next we estimate $|\R\cap\R'|$. As each record in $\R$ comes from connecting all the nodes in the steiner tree, we multiply all the weights in the edges in the steiner tree as the likelihood that this record is in $\R\cap\R'$. We aggregate all the likelihoods of all the records in $\R$ as an estimation of $|\R\cap\R'|$, such as summation. In this way, we can estimate the precision of the query result.


\dong{It seems impossible to estimate $|R'|$?}

Next we estimate the recall of the query result, which can be achieved by estimating $|\R'|$. To this end, we traverse the steiner tree in post-order. Suppose we are visiting a node $n$, whose parent is $p$. If $n$ is the foreign key, we set the cardinality of $p$ as the size $|n|$ of table $n$.... \dong{Seems hard to estimate the recall, even if the steiner tree is a 'line'.}


%Suppose that the edges in the maximum bipartite matching are $e_1, e_2, \cdots, e_n$ and the weights are $w_1, w_2, \cdots, w_n$. We use an aggregation function $Agg$ to combine all the weights, such as summation, and treat this as an estimation of the number of records $A$ and $B$ shared. As the number of records in the final view is determined by the number of foreign keys, thus we estimate the precision of the edge from $A$ to $B$ as $$\frac{Agg(e_1,e_2,\cdots,e_n)}{Noise(A.a)*|A.a|}$$ where $Noise(A.a)$ is an estimation of the noisiness of $A.a$, i.e., the ratio of irrelevant values in $A.a$. We multiply all the estimated precision of every edge in the steiner tree as our final estimation of the precision.






\subsection{Maximize Quality Gain with a Budget}\label{subsec:gain}

We first define a cleaning operation and its cost. We involve human to clean the data by asking them question. We ask two kinds of questions. The first one is ``\texttt{Does tuple t satisfy a predicate P?}'' and the second one is ``\texttt{Are tuple t and t' the same object?}''. Assume all the questions have the same cost. Then given a budget, we can only ask the human a limited number of questions.



As we only give an estimation of the query result quality, after the user cleans some data, our estimated result quality can decrease in some cases. This is because our estimation on the more clean data is more accurate than our original estimation, which can be over estimated the quality. We would not want this situation to happen. Thus we given three possible solutions to avoid this situation. \dong{Which sounds best?}


1) We can come up with a confidence of our estimation. Our original estimation can have a lower confidence (larger quality interval) and the updated estimation has a higher confidence (smaller quality interval). We can show the user the lower bound of the result quality, which must increase when the users clean some data.

2) We give a lower bound of the cleaning effort that the user must spend to make the estimated quality increase.

3) We generate a chain of questions for the users to answer so as to clean the data. We guarantee the estimated quality cannot go down at any time when the users clean the data using our plan.


\iffalse
We fist discuss As each arc in the query chain has 

To maximize the recall gain with the given budget, we can develop a dynamic programming method


In the path query, we assume the error can only happen in the evaluation of the predicate $P$ or in the tuple alignment within two tables. The minimum cleaning unit can be verifying a pair of tuples in the tuple chain or testifying whether a value satisfies the predicate $P$. For ease of presentation, we assume they both has constant cost $c$. In this section, we address the problem of generating the cleaning plan that maximizing the quality gain of the result with a given budget.


As the tuple chain of different records in the view can share the same tuple pair. 




Let $A_i$ and $B_i$ respectively be the foreign key and primary key in table $T_i$. Using the error-robust inclusion dependency, we can align (by the maximum bipartite matching) the values in $B_i$ and $A_{i+1}$ where $1\leq i\leq K-1$. Let $b^i$ and $a^{i+1}$ respectively be a value in $B_i$ and $A_{i+1}$. We use $S_{clean}(a_{i+1},b_i)$ to denote the cleanliness score between $a$ and $b$ and $S_{clean}(a_1,P)$ to denote the cleanliss score.


Let $a^{i+1}_1, a^{i+1}_2, \cdots, a^{i+1}_y$ and $b^{i}_1, b^{i}_2, \cdots, b^{i}_x$ respectively be the distinct values in  $A_{i+1}$ and $B_i$. Suppose $a^{i+1}$


The first thing we should make clear is what is the minimum cleaning unit. 

Let’s first focus on the presence errors. Actually we should first focus on 

Suppose there are three tables T1, T2 and T3 as shown in below.


There are FK-PK relations between T1 and T2 and T2 and T3. Note for each attribute we collapse all the rows by the same value. For example, in table 2, the cardinality of the distinct values of the red attributes are all 1 while the cadinality of the distinct values of the green attributes are m1, m2, and m3 respectively.


When there is no filter, then R1 will have n1+n4=m2 rows. R2 has n3=m1 rows and R3 has n2=m3 rows. Then we should choose the the maximum of 

\subsection{Improving the Precision}
Suppose that we have a cleanliness estimation on each edge. Then we can estimate the cleanliness of the record in the view as the product of the cleanliness on each edge.

\subsection{Improving the Recall}
Consider a view generated by 

\begin{table}
\centering
\begin{tabular}{|c|c|c|c|c|} \hline
Name & Age & Dept & School & student no \\\hline
Mike & 20  & CS   & ENG &  5200 \\\hline
Samuel & 26  & CS   & ENG &  5200 \\\hline
Albert & 39  & MATH & SCI &  4100 \\\hline
James  & 42  & BIO  & SCI &  4100 \\\hline
David  & 60  & EE   & ENG &  5200 \\\hline
Robert & 99  & EE   & ENG &  5200 \\\hline
\end{tabular}
\end{table}

\lstset{language=SQL}
\begin{lstlisting}
Select *
From Faculty, Dept_School, School
Where School.student_no > 4000
\end{lstlisting}





\subsection{A Cleanliness Model wrt Precision}\label{subsec:model:precision}


We first give the intuition of our cleanliness model wrt precision. As the result of a \emph{path query} is a view, we estimate a cleanliness score for each record in the view and utilize an aggregation function to combine the cleanliness scores of all the records. The aggregated score is the cleanliness of the result. As each record in the view is a ``chain'' of tuples from different tables, we can estimate the cleanliness of each part in the tuple chain and use their product as the cleanliness score of this record. Next we formally define our cleanliness model.


Assume there are $K$ tables, $T_1$, $\cdots$, $T_K$ in the \emph{path query} where the predicate $P$ applies on $T_1$. Let $R_1, R_2, \cdots, R_n$ be the records in the view. For each record $R_i$, suppose that the tuple chain is $(t^i_1, t^i_2, \cdots, t^i_m)$. For the tuple $t^i_1$ where the predicate $P$ applies, we estimate the cleanliness score as $S_{pred}(t^i_1, P)$. For each tuple pair $t^i_j$ and $t^i_k$ in the chain, we can estimate the cleanliness score as $S_{join}(t^i_j,t^i_k)$. This score is achieved either by the error-robust inclusion dependency as described in Section 3 or the cleanliness estimation in Section 5. Then the cleanliness score for the record $R_i$ is $C_{R_i}=S_{pred}(t^i_1,P)\prod_{i=1}^{m-1}S_{join}(t^i_j,t^i_k)$. Suppose we use the aggregation function $\mathbf{A}$ to combine the cleanliness scores of all the records. The quality of the view is $\mathbf{A}(C_{R_1}, C_{R_2}, \cdots, C_{R_n})$.


\fi



