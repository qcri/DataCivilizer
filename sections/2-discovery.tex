% !TEX root = ../main.tex
\section{Discovery}
\label{sec:discovery}

The goal of the data discovery module is to find relevant data among the
millions of datasets spread across many storage systems of modern organizations.
Suppose an analyst wants to answer the question: what is the monthly sales trend
by department? The analyst knows conceptually what data is needed to answer this
question (a table of sales, a table of departments, a table of products sold by
each department), but not which specific data sources (which relations in a
schema or what files in an HDFS deployment) contain such data. The typical
solution is to 1) ask an expert (if such a person is available), or to perform
manual exploration, inspecting datasets one by one (which is time-consuming and
prone to missing relevant data sources). We say this analyst is facing a data
discovery challenge.

The data discovery module narrows down the search of relevant data from the
thousands and millions of data sources to a handful of them that can then be fed
to stitching---that performs the final preparation before processing. Discovery
exports an API that can be used by users directly to search for relevant data,
e.g., schema search, similar content, etc, as well as by other modules in Data
Civilizer, such as stitching, to retrieve additional data that can be used, for
example, to enrich a table. Next we describe the general steps of data
discovery:

\subsection{Data Discovery Components}

The data discovery module consists of two conceptually different components that
collaborate with each other to build a skeleton of the linkage graph shown in
Figure~\ref{fig:arch}. The linkage graph is defined as $G=(V,E)$, where $E$ is a
multiset, as we permit different types of edges, \ie relationships, betweeen
$V$. To build this multigraph, discovery consists of two modules, the first one,
the \code{data profiler} is in charge of finding the $V$ in the graph, and the
second one, the \code{graph builder} is reponsible for finding $E$.

\begin{enumerate}
\item {\bf Data profiler.} We first accumulate knowledge of the data sources by
summarizing them into concise profiles. We can choose the granularity of those
profilers. In practice, we have found that building a profile per attribute is
sufficient for most use cases. A profile contains a signature, which is a
domain-dependent, compact representation of the original content. One example of
signature for numerical data is its distribution, and for textual data a vector
with the most significative terms. A profile also contains information about the
data cardinality, data type, and numerical ranges when it
applies~\cite{profiling_survey}.  

\item {\bf Graph builder.} The data profiler generates a set of $V$, and the
graph builder reponsibility is to find relationships among these, and represent
them as $E$, building the skeleton of the linkage graph. These relationships
help to navigate through the different sources ($V$). Examples of relationships
that the graph builder finds are \emph{content similarity} based on measuring
the similarity distance among the signatures that represent each $v$. Another
relationship is \emph{schema similarity} that captures the similarity of the
names of the different $v$. Other kind of relationships capture the hierarchical
relationship among $v$, for example, all attributes of a same relation are
connected, permitting quick navigation. Creating these relationships require in
general a pairwise comparison that would render the module too slow. We discuss
next some techniques we use to tame the scale.
\end{enumerate}

Once the linkage graph skeleton has been built, it can be accessed by a set of
data discovery primitives to explore and find relevant data. These primitives
are used by users and by other modules, such as stitching. In particular,
stitching also has write access to the linkage graph, that uses to add a new $e$
to the multiset of relationships, the expensive-to-compute FK-PK relationships.
Data discovery primitives can be composed into more complex data discovery
functions, and all of these combine through combinator operators to build
expressive discovery queries.  

\subsection{Data Profiler}

The profiling module of data discovery is responsible for computing signatures
for each attribute in the dataset, as well as data types, cardinalities and an
estimation of the \emph{cleanliness} of the data, \ie a measure of its data
quality. Signatures are type-dependent and must exhibit two properties: (i)
represent the attribute values in a compact way; and (ii) be compatible with a
similarity metric. 

Signatures can be of two types, numerical and textual. For numerical values the
module learns the probability distribution of the data and it also computes the
data range by estimating the median and inter-quantile range, \ie the $75p -
25p$. In the case of textual data, the module computes the TF-IDF vector for
each profile of data, capturing the terms that better represent the attribute.

\textbf{Fast cardinality and quantile estimation.} Due to the sheer volumes of
data, discovery relies heavily on sketches to estimate the cardinality of the
attributes as well as the quantiles in the case of numerical data. This avoids
ordering the data. More in general, this permits the module to compute the
profiles following a \emph{read-once} principle, saving memory and computation
time. By treating the input data as a read-once stream we also avoid putting too
much pressure on the original data sources.

\textbf{Data quality estimation.} Along with data cardinalities and quantiles,
that can help to identify candidate outliers, the profiling module also collects
information about common data errors, such as empty values. When safe, it
performs some denoising, \ie removing empty values, leading to better quality profiles, and then it
includes this \emph{cleanliness} information as part of the profile. This serves
as a preliminary step to other more-expensive cleaning operations performed by
other modules in data civilizer, such as stitching and the query processor \ref{sec:4,
sec:5}). 

\subsection{Graph Builder} The input of the graph builder is a set of profiles,
that are represented as nodes in a multigraph. The goal of the builder component
is to find content and schema similarity relationships among these nodes, which
are then represented as edges. In this way, the multigraph will have an edge of
type \emph{content-similarity} between nodes A and B if these are similar, and
one of type \emph{schema-similarity} if the schema names are similar too. Each
edge is scored with a weight that represents the strength of the relationship:
the particular meaning depends on the edge semantics.

Although this approach creates a logically complete graph, in practice it is
possible to define a minimum similarity threshold to prune edges. Building the
graph requires a pair-wise operation among all attributes in the
entire dataset, a $N^2$ operation. To scale this process, we are introducing
approximate nearest neighbor techniques based on locality sensitive hashing \cite{DBLP:conf/compgeom/DatarIIM04}
that helps to scale the process. In addition, we have built a distributed
prototype that can ingest data and compute signatures and search paths in
parallel.


