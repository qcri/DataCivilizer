% !TEX root = ../main.tex
\section{Data Stitcher}
\label{sec:stitching}

\newcommand{\eind}{error-robust inclusion dependency\xspace}
\newcommand{\ind}{inclusion dependency\xspace}
\newcommand{\R}{\ensuremath{R}\xspace}
\renewcommand{\S}{\ensuremath{S}\xspace}
\newcommand{\X}{\ensuremath{X}\xspace}
\newcommand{\Y}{\ensuremath{Y}\xspace}
\newcommand{\RX}{\ensuremath{\R[\X]}\xspace}
\newcommand{\SY}{\ensuremath{\S[\Y]}\xspace}
\newcommand{\IND}{IND\ensuremath{(\X,\Y)}\xspace}
\newcommand{\EIND}{\texttt{EIND}\ensuremath{(\RX,\SY)}\xspace}
\newcommand\subsetsim{\mathrel{%
  \ooalign{\raise0.3ex\hbox{$\subset$}\cr\hidewidth\raise-0.6ex\hbox{\scalebox{0.8}{$\sim$}}\hidewidth\cr}}}
\newcommand{\G}{\ensuremath{G}\xspace}
\newcommand{\E}{\ensuremath{E}\xspace}
\newcommand{\U}{\ensuremath{U}\xspace}
\newcommand{\V}{\ensuremath{V}\xspace}

\newtheorem{theorem}{Theorem}
\newtheorem{example}{Example}
\newtheorem{definition}{Definition}
\newtheorem{proposition}{Proposition}
\newtheorem{lemma}{Lemma}
\newtheorem{corollary}{Corollary}




The linkages between different data can be extremely helpful for the users to find the interesting data. Data stitching links all the data together and adds these linkage into the graph. Among all the data linkages, the primary key and foreign key (PK-FK) relationship is one of the most important. Data stitching first utilizes the inclusion dependency to find the candidate PK-FK relationship and then refine the candidates by advanced techniques. However, as we all known, data in the wild is rather dirty. The dirty data contaminates the inclusion dependency in two ways: make fewer keys overlap and make keys not exactly match. To tolerate errors, we extend the traditional inclusion dependency by both key coverage and text similarity and propose the \emph{\eind}. 

Note it is time consuming to find the \eind. To address this issue, data stitching leverage the data profiling results from data discovery to reduce the search space. Moreover, as the advanced techniques that refine the PK-FK candidates are computation intensive, we only apply them after the users identified his interesting data.

Data stitching also estimates the cleanliness of the linkage in the graph, which can help curate the ploystore query.

%1. Limit the search space

%2. Tolerate errors in includency dependency

%3. Estimate the cleanings in join path

%4. Querying the graph


\subsection{Error-Robust Inclusion Dependency}\label{subsec:eind}


Foreign key is one of the most important schema information in managing and using data, which is typically missing in real world. The foreign key and primary key relations are usually identified by inclusion dependency. However, data in the wild is full of errors, such as inconsistence and different formats. This yields to the requirement of error tolerating in inclusion dependency. We observe that the errors in data can contaminate the inclusion dependency in two ways. First, they make the corresponding primary keys and foreign keys not match exactly. Second, they make the foreign keys not all covered by the primary keys. To address these issues, we design an error-robust inclusion dependency which enhances the traditional inclusion dependency with value coverage and text similarity. Next we give the details.


%The main idea behind the error-robust inclusion dependency is that we use the text similarity between the keys to estimate the possibility of they represent the same object. As each distinct foreign key can only be mapped to one primary key and vice versa, we further use the maximum weighted bipartite matching to align the primary keys and the foreign keys. Note it is challenging to efficiently find all the error-robust inclusion dependencies within a bunch of tables, as the input tables can be large and both finding the maximum weighted bipartite matching and calculating the text similarities between all the primary keys and foreign keys are time consuming.



Given two projections \RX and \SY, we build a weighted bipartite $\G=((\U,\V),\E)$ based on the projections. There is a bijection between the vertexes in \U and the distinct instances in \RX, i.e., each vertex $u$ in $\U$ maps to a distinct instance in \RX and vice versa. This also applies to the vertex set \V and the distinct instances in \SY. For any vertexes $u\in\U$ and $v\in\V$, there is a weighted edge $e=(u,v)\in\E$ where the weight $w_e$ is defined by the text similarity between their corresponding instance values, such as Jaccard similarity and edit similarity. Let $\E_{max}\subseteq\E$ be the \emph{maximum weighted bipartite matching} of \G and $\EIND=\frac{\sum_{e\in\E_{max}}w_e}{|\U|}$. Then \EIND is proportional to the chance of an inclusion dependency from $\RX$ to $\SY$. Given an error-tolerating threshold $\delta$, we formally define the \eind as follows.
%is a set of edges $\E_{max}$ which satisfies (1) any two edges in it share no common vertex and (2) the sum of edge weights is maximum. 


\begin{definition}[Error-Robust Inclusion Dependency]
Given two projections \RX and \SY on relational tables and an error-tolerating threshold $\delta$, there is an error-robust inclusion dependency from \X to \Y when $\EIND\geq\delta$.
\end{definition}

\iffalse
\dong{To add an example here.}
\fi


Note the value domain of $\EIND$ is $[0,1]$. there is an \emph{exact} inclusion dependency among the two projections if and only if $\EIND=1$. When the projections cross multiple fields, we can utilizes different text similarity functions on different fields. As it requires all the fields to be match in inclusion dependency, we combine the text similarities in different fields by multiplying them. For example, given two instances \textsf{(SIGMOD Conference, Sam Madden, San Francisco)} and \textsf{(SIGMOD Conference 2016, Samuel Madden, San Francisco)}. Suppose that we uses Jaccard similarity to evaluate the first field and edit similarity to evaluate the second field. Then we can combine the text similarities as $\frac{2}{3}*\frac{7}{10}*1=0.47$. The score \EIND can also serve as the cleanliness estimation of the linkage between \RX and \SY	.

\subsection{Refine Candidate FK-PK Relationships}

The error-robust inclusion dependency gives us a bunch of candidate foreign key and primary key (PK-FK) relationships. We can apply existing machine learning method to remove the false positives from the results. 




\iffalse
\subsection{Query the Join Graph}\label{subsec:query}

Once the join graph is constructed, the users can query it in various way. Among them, one of the most important one is to.

The essential way to query the join graph is taking several vertexes and find an subgraph containing all the query vertexes.

The user can specify several attributes which compose a \emph{query schema}. 

Given a collection of attributes, we aim to find a subgraph that contains all the corresponding vertexes. 

%\subsection{Cleanliness Estimation}
%finding ind in a scalable way
%column de-duplication
\fi