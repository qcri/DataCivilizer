% !TEX root = ../main.tex

\begin{abstract}
In large companies, it is often challenging for users to find relevant data for  specific tasks, from using simple keyword queries to more complex structured queries, since the data is often scattered, inconsistent, and frequently updated. In order to decrease the ``grunt work'' needed to facilitate the analysis of data in the wild, we present an end-to-end data management system, named \dcv. 
%
It has a {\em data discovery} module to identify the data that is relevant to user tasks.
It also contains a {\em linkage graph computation} module that links all discovered relevant data on demand.
In addition, its {\em query processing} module is built on top of a polystore architecture to compute a result for a given user task, which also integrates data cleaning operations.
In practice, different tasks might invoke the above modules in different orders, and might be iterative.
To cope with this, we implement a {\em workflow} engine to enable the arbitrary composition of different modules, as well as to handle data updates.
We have deployed our preliminary \dcv system in two institutions, MIT and Merck. The users gave us positive feedback and indicated that the system indeed shortened their time and effort to find, prepare and analyze the data.
%In this paper, we present an end-to-end data management system, named \dcv, whose main purpose is to decrease the  ``grunt work''\mourad{I prefer http://dictionary.cambridge.org/dictionary/english/grunt-work.} needed to facilitate the 
%``mung work''  for  analyzing analysis of data in the wild. 
% To solve these problems, \dcv has a data discovery module to help users identify their interesting data. 
%We also develop a data stitching module to link all the relevant data for a better usage. 
%We build the system on top of a polystore architecture and integrate the data cleaning operations in the query plan. 
%In addition, as the data preprocessing conducted by the users are often repetitive, we design a workflow engine to enable the composition and the iteration over the above modules as well the handling of updates. We deployed our preliminary \dcv system on two institutions, MIT and Merck. The users gave us positive feedbacks and indicated that the system indeed shortened their time and effort to prepare and analyze the data.
\end{abstract}