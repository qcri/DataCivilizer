% !TEX root = ../main.tex
\section{Data Civilizer in the Wild}
\label{sec:wild}

We have deployed the current prototype of Data Civilizer inside two organizations from two different domains: Merck and the MIT Data warehouse. Table~\ref{tab:dataCivInTheWild} depicts the different size of each organization and the existing use cases that were mentioned by the data scientists and database users of the organization. In the following, we report on requirements from each organization and how Data Civilizer covered them. Furthermore, we report on the experience of the domain experts with our system and how Data Civilizer's general framework of opertions was able to cover the user requirements inside each organization.

\begin{table}
\caption{Deployment environments of Data Civilizer}\label{tab:dataCivInTheWild}
\begin{tabular}{|l|l|l|l|}
\hline
Organization & \# databases & \# tables & Use cases\\
MIT DWH & & & \\
Merck & & & \\
%Doha Traffic Signals & 9 & millions & ??? \\
\hline
\end{tabular}
\end{table}

\subsection{MIT Data warehouse}

The MIT Data Warehouse contains XXX tables with information about \ldots.

The users of this data warehouse are primarily using the database for \ldots
Challenging tasks that require a data discovery and munging module on top of the warehouse are \ldots

We deployed Data Civilizer as a \ldots service and allowed the users to interact with the system through the graphical interface. 
Data Civilizer was able to \ldots

\subsection{Merck}



In summary, we can say that specific data discovery and munging tasks are prevalent depending on the domain.. While at Merck the user 
