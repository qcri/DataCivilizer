% !TEX root = ../main.tex
\section{Data Civilizer in the Wild}
\label{sec:wild}

We have built an initial prototype of \dcv containing the data discovery module
and the linkage graph builder as discussed in Section~\ref{sec:discovery} and~\ref{sec:stitching}. Ultimately, we will integrate \dcv with
\texttt{BigDawg}, our implementation of a Polystore, to realize the end-to-end
system described in this paper.  We have been working closely with end users to
adapt our prototype to relevant real world problems.  In particular, we have deployed
a preliminary prototype of discovery in two different organizations The MIT Data
warehouse (MIT DWH) is a group at MIT responsible for building and maintaining a
data warehouse that integrates data from multiple source systems. Merck, a big
pharmaceutical company, manages large volumes of data, which are handled by
different storage systems.  In the following, we first outline the current state
of the system and the modules that are currently deployed.  We then provide
details on each organizations' requirements, and how we are using \dcv to help
them.

\subsection{Current \titledcv Prototype} The current prototype runs as a
server and can access data that resides in a file system or one or more
databases.  Clients  connect to \dcv through a RESTful API.  The current
deployment of \dcv  includes both  the graph builder with the error-robust
inclusion dependency discovery from Section~\ref{sec:stitching} and the
discovery component described in Section~\ref{sec:discovery}.  Thus a user is
able to submit queries that consist of attribute names, attribute values, or
tables, and receive different types of results based on the selected
similarity function. The similarity function depends on the specific use case,
which we will discuss based on our two use cases MIT and Merck.  We plan to
demo our \dcv prototype at the conference.

\subsection{MIT Data warehouse}

One of the key tasks of the MIT DWH team is to assist its customers --
generally MIT administrators -- to answer any of their questions about MIT
data. For example, staff usually want to create reports, for which they need
access to various kinds of data. The warehouse contains around 1TB of data
spread across approximately 3K tables.

In their current workflow, a
DWH customer presents a question, which a DWH team
members by manually searching for tables containing relevant data. Once they have determined
the tables of interest, they create a view that is accessed by the customer to
solve the question at hand. Below are some of the common use cases we have
encountered:

\mypar{Fill in virtual schema} When a customer arrives with a question such as:
\emph{I need to create a report with the \textbf{gender distribution of the
faculty per department and year}}, the data warehouse personnel can use \dcv to
find all the tables that contain schema names similar to the attributes exposed
by the query, \eg gender, faculty name, department, and year.

\mypar{Table redundancy} Multiple views are created for different customers.
Many of them contain very similar data, as multiple customers are
interested in similar items. To reduce the redundancy of data, \dcv helps
detect complementary as well as repeated sources. This sheds light on the status
of the warehouse and helps to maintain it tidy and minimal.

Our prototype is deployed in an Amazon EC2 instance managed by the MIT DWH team
with access to their data. We have been working with them for 3 months,
iterating over priorities and learning about the problems they are facing.

In the future, we aim to deploy the entire \dcv system to enable running queries
directly over the hundreds of data sources, by using the polystore query
processing and curation module and linkage graph builder to create the necessary
views on demand. 

\subsection{Merck}

Merck is a large pharmaceutical company that manages massive volumes of data spread
across around 4K databases in addition to several data lakes. 
%The use cases are varied, however there is a common case. 
One of the data assets of any pharmaceutical company are internal databases of
chemical compounds and structures. Usually, these are more valuable when
integrated with external, well-known and curated databases, such as
PubChem~\cite{pubchem}, ChEMBL~\cite{ChEMBL}, or DrugBank~\cite{DrugBank}. We
describe two common use cases that occur in this context:

\mypar{Enrich data} One of the reasons for the existence of multiple chemical
databases is that each puts an emphasis on different information. Analysts
typically face situations in which they are interested in a set of attributes
that are spread across different tables on different databases. \dcv helps to
detect such attributes and bring them together \emph{on-demand}  to serve the
users' purpose.

\mypar{Identify entities} One single chemical entity may be referred to with
different identifiers and formats in different databases. Chemical identifiers
have been a subject of research in the bioinformatics community: multiple
different formats have been proposed with different properties according to the
scenario. \dcv helps them on this task by discovering datasets that contain
schemas with the entities of interest.

We have been collaborating with 4 engineers and bioinformatics experts at Merck
during the last 2 months. During this period we have learned about their use
cases and we have used discovery on chemical databases that are publicly
available. In the future, we plan to integrate internal datasets too, with the
goal of evaluating \dcv with more varied datasets.

