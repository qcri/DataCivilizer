% !TEX root = ../main.tex
\section{Discovery}
\label{sec:discovery}

The data discovery module's goal is to find relevant data from perhaps hundreds
to thousands of data sets  spread across the different storage systems of an
enterprise, using the linkage graph.
% to provide answers efficiently.

Current solutions to finding relevant data include asking an expert (if one is
available) or  performing manual exploration by inspecting data sets one by one.
Obviously, both approaches are time-consuming and prone to missing relevant data
sources. Recently, there have been efforts to automate this process, some
examples are Goods~\cite{DBLP:conf/sigmod/HalevyKNOPRW16} and
InfoGather~\cite{DBLP:conf/sigmod/YakoutGCC12}. Goods permits users to inspect
datasets and find others similar to datasets of interest. Infogather permits to
extend the attributes of a dataset  with attributes from other
datasets. These systems are designed to solve these particular use cases, and so
their indexes are built for this specific purpose. In
contrast, our discovery module aims to support general \emph{discovery queries} that
can then be used for unanticipated discovery needs. Hence, the discovery module
uses a linkage graph (Section \ref{subsec:graphbuild}) to permit a broader range
of queries. 

%Mourad: A bit handwaving!

%\srm{I don't understand the
%following sentence, vis-a-vis the Goods paper, which seems to be about
%supporting data discovery for a bunch of use cases, without one specific
%use-case in mind.}\mourad{I agree with Sam; after skimming through the paper, I
%see Goods as a serious competitor to DC, at least on the discovery side.}
%
%Unlike these systems that are designed with a specific use-case in mind, our
%discovery module aims to collect enough information about the data sets and to
%expose them to upper layers of \dcv, allowing the orchestration of complex query
%workflows that integrate a variety of cleaning, linking, and querying
%operations.

%execution, which is the ultimate goal.  

The data discovery module narrows down the search for relevant data from
the entire space of available data sources to a handful, on which finer-grained
relationships can be computed efficiently. \dcv discovery supports a
user-level API to search for relevant data using a variety of techniques such as
searching for tables with specific substrings in the schemas ({\it schema
search}) or table values ({\it content search}). 
%Discovery is extensible; an engineer can extend the supported relationships to
%cover other relevant use cases.

\subsection{Discovery Queries}
\label{subsec:api}

Users can submit
\emph{discovery queries} to find data sets whose schemas are relevant to the
task at hand. For example, to find all data sets that contain names of employees
in a company, the user issues a discovery query that searches for all attributes
referring to an employee or a name. The user can then write a SQL query to filter
from a list. Here are some examples of functions that the discovery module
provides to users;  each of these queries can be answered using data collected
as a part of profiling as well as the information  in the
linkage graph:
%\srm{I think a little more detail is
%needed here describing how the profiles built above are used to answer these
%queries.  As it is the two subsections don't seem well connected -- why are
%profiles needed?} 

\begin{itemize}
\item \textbf{Fill-in schema}. Given a set of names, this function retrieves 
tables or groups of tables that contain the desired attributes. The core
of the operation consists of finding tables with similar attribute names to the
provided ones, using the keyword index.


\item \textbf{Extend attribute.} This is similar to the extend operator of
Octopus~\cite{DBLP:journals/pvldb/CafarellaHK09}, or the ABA operation of
Infogather~\cite{DBLP:conf/sigmod/YakoutGCC12}. Using this function users can
extend tables of interest with additional attributes. The core of the operation
consists of finding matches to the current table using the linkage graph, and then retrieving 
attributes that do not appear in this table.

\item \textbf{Subsumption relationships.} Given some reference table, this
function provides a list of tables or groups of tables that have some form of
subsumption relationship (\ie are contained in or contain) with respect to the
reference table;  this can be done directly from the linkage graph.


\item \textbf{Similarity and lookup.} Discovery can also be used to find schemas
that contain data similar to some provided schema (attribute or table), as well
as to find schemas that contain certain values, \ie keyword search for textual
data and range search for numerical data.  


\end{itemize}

Discovery works similarly to an information retrieval system: schema retrieval queries
do not return a specific answer, but instead return a ranked set of results. We
are currently exploring several ranking options, and a model that decouples
discovery query processing from ranking evaluation seems the most promising.

%\srm{Any suggestions about how to do this?}

