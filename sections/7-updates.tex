% !TEX root = ../main.tex
\vspace{-.1em}
\section{Updates}
\label{sec:updates}

Real-world data is rarely static and hence we need some way to deal with new datasets and updates. We consider three types of updates.

\stitle{(1) Insertions/deletions on source tables.} This happens when there is a
change to a table, \eg the insertion of a new procurement record in the MIT Data
Warehouse. This may also happen when data sources are cleaned (see
Section~\ref{sec:curating}).


\stitle{(2) Replacement of source tables.} Large companies typically rely on
both internal and external information to build their knowledge bases. For
instance, Merck collects published standard medical names from the World Health
Organization (WHO) to help construct their own ontology. These data sources are
updated periodically by WHO.  Sometimes, even the format may be changed, e.g.,
from a JSON file to a CSV file.


\stitle{(3) Updating MVs.} MVs might be created based on other MVs.  Since
new data can arrive at any time, and cleaning can be done at any time to any record, MVs may need to be
updated.


%In response to the above three types of updates, 
\dcv uses three corresponding strategies to cater to the above updates.

\stitle{(i) MV maintenance.} \dcv will need to incrementally propagate updates through the data curation pipeline to update downstream MVs, as well as from intermediate results that are cleaned back to data sources, if possible.
To perform this propagation, we plan to leverage the techniques in DBRx~\cite{DBLP:conf/sigmod/ChalamallaIOP14}, a system developed by QCRI and Waterloo.
In some cases, the human effort to update and clean MVs may be daunting, and automatic methods may fail. In such cases, MVs should be discarded rather than updated. 

\stitle{(ii) Provenance management.}  To perform update propagation, \dcv will need to keep track of the relationships between data sets (their provenance). \dcv will leverage Decibel~\cite{DBLP:journals/pvldb/MaddoxGEMPD16}, a system developed at MIT for this purpose.

\stitle{(iii) Incremental graph updates.}  As data changes, the linkage graph and profiles will also need to be updated.  To support incremental updates  to profiles we  use a simple reference counting scheme, where each keyword has a count associated with it that is decremented when a word is removed.  A word
is removed from the profile when its count reaches zero.  To support updates to the graph, rather than recomputing relationships every time a table changes, we 
maintain an estimate of the fraction of rows in a column that have changed since we last updated its node and edges in the graph, and re-index a column when
this estimate goes above a set threshold.

%As data appears incrementally, we need an incremental inclusion dependency algorithm. Extending our algorithms in Section~\ref{subsec:eind} is a project we are currently working on.

