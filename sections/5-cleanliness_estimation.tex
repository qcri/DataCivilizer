% !TEX root = ../main.tex
\section{Cleanliness Estimation}
\label{sec:cleanliness}

After running the  discovery and stitching, a user identifies a virtual relation (or view) to query.  This includes identifying the subset of each source table of interest.  In other words, the user has defined the join path using stitching and the predicates that subset the tables in some other way.   Directly querying this view or the base tables without any cleaning entails two risks: (i) Missing results (false positives) and (ii) Reporting wrong results (false negatives). For example,  if one has a data value, ?New Yark? in a column city, and wants to perform a join on city, then ?New Yark? should be corrected to ?New York?, otherwise the join will miss some results (False negatives). Now if the column contains wrong but valid values, then these will ?wrongly? be joined. 

In addition to the user query, we assume the following input are given, namely sizes of base relations, cost of cleaning a cell (could be per relation, per column, and per tuple or we could assume one unit cost for any cell),  accuracy metrics per column which gives an estimate for the percentage of the column which is erroneous,  user?s goal for cleanliness P in terms of percentage of erroneous data,  and cleaning budget M which is the budget the user is ready to spend on cleaning.

Since cleaning usually entails human efforts, we need a (i) way to estimate the cleanliness of the query results and (ii) how is it affected by the type of operations and (iii) how it can be changed by applying cleaning (which entails human efforts) on specific data values. Assuming SPJA (selection, projection, join, aggregation, and grouping) queries, each of the operation will have a different effect on the cleanliness of the results. More specifically,  with selection, the ?amount? of errors will go down, with projection,  the amount of errors will go down, error may even disappear if projection drops erroneous columns and keep clean ones, with join, the amount of errors may increase, and with grouping and aggregate, the amount of errors may decrease. 

{Mourad: It will be good if we can devise some sort of a formula for estimating cleanliness. Maybe we should also move the text on ?disorder metric? here.}