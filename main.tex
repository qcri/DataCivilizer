\documentclass{sig-alternate-05-2015}

\usepackage{graphicx}
\usepackage{balance}
\usepackage{caption}
\usepackage{hyperref}
\usepackage{url}
\usepackage{microtype}
\usepackage{paralist}
\usepackage{booktabs}
\usepackage{amssymb}
\usepackage{amsmath}
\usepackage{mathtools}
\usepackage{listings}
\usepackage{cite}
\usepackage{subscript}
\usepackage{enumitem}
\usepackage{xspace}
\usepackage{amssymb}

\let\proof\relax
\let\endproof\relax

\usepackage{amsthm}
\usepackage{float}
\usepackage[algoruled, linesnumbered, vlined]{algorithm2e}
\usepackage[utf8]{inputenc}

\usepackage{inconsolata}

\usepackage{prp-macros}

\newcommand*\samethanks[1][\value{footnote}]{\footnotemark[#1]}
\newtheorem*{example*}{Example}

\lstset{ %
   belowskip=-2em,
   backgroundcolor=\color{white},   % choose the background color; you must
   basicstyle=\scriptsize\ttfamily,        % the size of the fonts that are used for the code
   breakatwhitespace=false,         % sets if automatic breaks should only happen at whitespace
   breaklines=true,                 % sets automatic line breaking
   captionpos=b,                    % sets the caption-position to bottom
   commentstyle=\color{Brown},    % comment style
   deletekeywords={...},            % if you want to delete keywords from the given language
   escapeinside={\%*}{*)},          % if you want to add LaTeX within your code
   extendedchars=true,              % lets you use non-ASCII characters; for 8-bits encodings only, does not work with UTF-8
   frame=,                    % adds a frame around the code
   keepspaces=true,                 % keeps spaces in text, useful for keeping indentation of code (possibly needs columns=flexible)
   keywordstyle=\color{MidnightBlue},       % keyword style
   language=,                 % the language of the code
   morekeywords={*,range,var},            % if you want to add more keywords to the set
   numbers=none,                    % where to put the line-numbers; possible values are (none, left, right)
   numbersep=5pt,                   % how far the line-numbers are from the code
   numberstyle=\tiny\color{gray}, % the style that is used for the line-numbers
   rulecolor=\color{black},         % if not set, the frame-color may be changed on line-breaks within not-black text (e.g. comments (green here))
   showspaces=false,                % show spaces everywhere adding particular underscores; it overrides 'showstringspaces'
   showstringspaces=false,          % underline spaces within strings only
   showtabs=false,                  % show tabs within strings adding particular underscores
   stepnumber=1,                    % the step between two line-numbers. If it's 1, each line will be numbered
   stringstyle=\color{RedOrange},     % string literal style
   tabsize=2,                       % sets default tabsize to 2 spaces
   title=\lstname,                   % show the filename of files included with \lstinputlisting; also try caption instead of title
   moredelim=[is][\color{blue}\bfseries\underbar]{@}{@},
}

\lstdefinestyle{myCSharp}{
   language={[Sharp]C},
   basicstyle=\scriptsize,
   belowskip=-2em,
   backgroundcolor=\color{white},   % choose the background color; you must
   basicstyle=\scriptsize\ttfamily,        % the size of the fonts that are used for the code
   breakatwhitespace=false,         % sets if automatic breaks should only happen at whitespace
   breaklines=true,                 % sets automatic line breaking
   captionpos=b,                    % sets the caption-position to bottom
   commentstyle=\color{Brown},    % comment style
   deletekeywords={...},            % if you want to delete keywords from the given language
   escapeinside={\%*}{*)},          % if you want to add LaTeX within your code
   extendedchars=true,              % lets you use non-ASCII characters; for 8-bits encodings only, does not work with UTF-8
   frame=,                    % adds a frame around the code
   keepspaces=true,                 % keeps spaces in text, useful for keeping indentation of code (possibly needs columns=flexible)
   keywordstyle=\color{MidnightBlue},       % keyword style
   morekeywords={var},            % if you want to add more keywords to the set
   numbers=none,                    % where to put the line-numbers; possible values are (none, left, right)
   numbersep=5pt,                   % how far the line-numbers are from the code
   numberstyle=\tiny\color{gray}, % the style that is used for the line-numbers
   rulecolor=\color{black},         % if not set, the frame-color may be changed on line-breaks within not-black text (e.g. comments (green here))
   showspaces=false,                % show spaces everywhere adding particular underscores; it overrides 'showstringspaces'
   showstringspaces=false,          % underline spaces within strings only
   showtabs=false,                  % show tabs within strings adding particular underscores
   stepnumber=1,                    % the step between two line-numbers. If it's 1, each line will be numbered
   stringstyle=\color{RedOrange},     % string literal style
   tabsize=2,                       % sets default tabsize to 2 spaces
   title=\lstname,                   % show the filename of files included with \lstinputlisting; also try caption instead of title
   moredelim=[is][\color{blue}\bfseries\underbar]{@}{@},
}

\lstdefinestyle{mySQL}{
   language={SQL},
   basicstyle=\scriptsize,
   belowskip=-2em,
   backgroundcolor=\color{white},   % choose the background color; you must
   basicstyle=\scriptsize\ttfamily,        % the size of the fonts that are used for the code
   breakatwhitespace=false,         % sets if automatic breaks should only happen at whitespace
   breaklines=true,                 % sets automatic line breaking
   captionpos=b,                    % sets the caption-position to bottom
   commentstyle=\color{Brown},    % comment style
   deletekeywords={...},            % if you want to delete keywords from the given language
   escapeinside={\%*}{*)},          % if you want to add LaTeX within your code
   extendedchars=true,              % lets you use non-ASCII characters; for 8-bits encodings only, does not work with UTF-8
   frame=,                    % adds a frame around the code
   keepspaces=true,                 % keeps spaces in text, useful for keeping indentation of code (possibly needs columns=flexible)
   keywordstyle=\color{MidnightBlue},       % keyword style
   morekeywords={var},            % if you want to add more keywords to the set
   numbers=none,                    % where to put the line-numbers; possible values are (none, left, right)
   numbersep=5pt,                   % how far the line-numbers are from the code
   numberstyle=\tiny\color{gray}, % the style that is used for the line-numbers
   rulecolor=\color{black},         % if not set, the frame-color may be changed on line-breaks within not-black text (e.g. comments (green here))
   showspaces=false,                % show spaces everywhere adding particular underscores; it overrides 'showstringspaces'
   showstringspaces=false,          % underline spaces within strings only
   showtabs=false,                  % show tabs within strings adding particular underscores
   stepnumber=1,                    % the step between two line-numbers. If it's 1, each line will be numbered
   stringstyle=\color{RedOrange},     % string literal style
   tabsize=2,                       % sets default tabsize to 2 spaces
   title=\lstname,                   % show the filename of files included with \lstinputlisting; also try caption instead of title
   moredelim=[is][\color{blue}\bfseries\underbar]{@}{@},
}


\graphicspath{{images/}}

\renewcommand{\labelitemi}{$\circ$}
\renewcommand{\labelitemii}{$-$}

\captionsetup[figure]{singlelinecheck=false, margin=0pt, font={sf,small},
  labelfont=bf, justification=centering}
\captionsetup[table]{singlelinecheck=false, margin=0pt, font={sf,small},
  labelfont=bf, justification=centering}
\captionsetup[lstlisting]{singlelinecheck=false, margin=0pt, font={sf,small},
  labelfont=bf, justification=centering}

\setlength{\textfloatsep}{5pt}
\setlength{\dbltextfloatsep}{5pt}
\setlength{\abovecaptionskip}{5pt}
\setlength{\floatsep}{5pt}


\renewcommand\AlCapFnt{\small\sffamily}

\def\Snospace~{\S{}}
\renewcommand*\sectionautorefname{\Snospace}
\renewcommand*\subsectionautorefname{\Snospace}
\renewcommand*\subsubsectionautorefname{\Snospace}
\newcommand{\corollaryautorefname}{Corollary}
\newcommand{\propositionautorefname}{Prop.}
\newcommand{\definitionautorefname}{Def.}
\renewcommand{\subsectionautorefname}{\Snospace}
\renewcommand{\figureautorefname}{Fig.}
\renewcommand{\equationautorefname}{Eq.}
\newcommand{\exampleautorefname}{Example}
\newcommand{\problemautorefname}{Problem}
\renewcommand{\algorithmautorefname}{Alg.}

\DeclareMathOperator*{\argmax}{arg\,max}
\DeclareMathOperator*{\argmin}{arg\,min}

\newcommand\mycommfont[1]{\footnotesize\rmfamily\textcolor{black}{#1}}
\SetCommentSty{mycommfont}

\newcommand{\myparnospace}[1]{\noindent \textbf{#1}}

\newcommand{\sys}{\textsc{Saber}\xspace}
\newcommand{\saber}{\sys}
\newcommand{\Saber}{\sys}
\newcommand{\qtsize}{$\varphi$\xspace}

\newcommand{\Sbs}{Stream batch size\xspace}
\newcommand{\sbs}{stream batch size\xspace}

\newcommand{\notemw}[1]{\textcolor{orange}{{\bf MW: {#1}}}}
\newcommand{\notealw}[1]{\note{\color{BlueGreen}[\textsc{ALW:} #1]}}

\newcommand{\CM}[1]{\textsf{CM\textsubscript{#1}}\xspace}
\newcommand{\SG}[1]{\textsf{SG\textsubscript{#1}}\xspace}
\newcommand{\LRB}[1]{\textsf{LRB\textsubscript{#1}}\xspace}
\newcommand{\dong}[1]{\textcolor{red}{\bf Dong: {#1}}}
\newcommand{\srm}[1]{\textcolor{red}{\bf Sam: {#1}}}

\newcommand{\eat}[1]{}

\renewcommand{\ldots}{\ifmmode\mathinner{\ldotp\kern-0.1em\ldotp\kern-0.1em\ldotp}\else.\kern-0.13em.\kern-0.13em.\fi}

\renewcommand*{\UrlFont}{\ttfamily\smaller\relax}
\newcommand{\zia}[1]{\textcolor{blue}{Zia: #1}}
\newcommand{\nan}[1]{\textcolor{magenta}{Nan: #1}}
\newcommand{\sibo}[1]{\textcolor{violet}{Sibo: #1}}

\newcommand{\mourad}[1]{\footnote{\textcolor{magenta}{Mourad: #1}}}

\newcommand{\stab}{\vspace{1.2ex}\noindent}
\newcommand{\sstab}{\rule{0pt}{8pt}\\[-2.2ex]}
\newcommand{\stitle}[1]{\vspace{1ex}\noindent{\bf #1}}
\newcommand{\etitle}[1]{\vspace{0.8ex}\noindent{\underline{\em #1}}}

\newcommand{\dcv}{\textsc{Data Civilizer}\xspace}


\begin{document}

% ****************** TITLE ****************************************

\title{The Data Civilizer System}

% ****************** AUTHORS **************************************

\numberofauthors{1}


\author{
 \alignauthor
}


\maketitle

% !TEX root = ../main.tex

\begin{abstract}
In large companies, it is usually hard for users to find the relevant data for their specific tasks, from simple keyword queries to complicated structured queries, since the data is often scattered everywhere, inconsistent, and updated frequently. In order to decrease the ``grunt work'' needed to facilitate the analysis of data in the wild, we present an end-to-end data management system, named \dcv. 
%
It has a {\em data discovery} module to identify the data that is relevant to user tasks.
It also contains a {\em join path computation} module that links all discovered relevant data on demand.
In addition, its {\em query processing} module is built on top of a polystore architecture to compute a result for a given user task, which also integrates data cleaning operations.
In practice, different tasks might invoke the above modules in different orders, and might be iterative.
To cope with this, we implement a {\em workflow} engine to enable the arbitrary composition of different modules, as well as handling data updates.
We have deployed our preliminary \dcv system on two institutions, MIT and Merck. The users gave us positive feedbacks and indicated that the system indeed shortened their time and effort to prepare and analyze the data.
%In this paper, we present an end-to-end data management system, named \dcv, whose main purpose is to decrease the  ``grunt work''\mourad{I prefer http://dictionary.cambridge.org/dictionary/english/grunt-work.} needed to facilitate the 
%``mung work''  for  analyzing analysis of data in the wild. 
% To solve these problems, \dcv has a data discovery module to help users identify their interesting data. 
%We also develop a data stitching module to link all the relevant data for a better usage. 
%We build the system on top of a polystore architecture and integrate the data cleaning operations in the query plan. 
%In addition, as the data preprocessing conducted by the users are often repetitive, we design a workflow engine to enable the composition and the iteration over the above modules as well the handling of updates. We deployed our preliminary \dcv system on two institutions, MIT and Merck. The users gave us positive feedbacks and indicated that the system indeed shortened their time and effort to prepare and analyze the data.
\end{abstract}
% !TEX root = ../main.tex
\section{Introduction}
\label{introduction}

An often-cited statistic is that data scientists spend 80\% of their time finding, preparing, integrating and cleaning data sets. The remaining 20\% is spent doing the desired analytic tasks. In practice, 80\% may be a lower bound; for example one data officer, Mark Schreiber of Merck, a large pharmaceutical company, estimates that data scientists in Merck spend 98\% of their time on ``grunt work'' and only one hour per week on ``useful work''.

In this paper, we present \dcv, a system we are building at MIT, QCRI, Waterloo, and TU Berlin, whose main purpose is to decrease the ``grunt work factor'' by helping data scientists to 
(i)~quickly {\it discover} data sets of interest from large numbers of tables;
(ii)~{\it link} relevant data sets; %Mourad:Well link and connect mean the same thing!?
(iii)~{\it compute} answers from the disparate data stores that host the discovered data sets;
(iv)~{\it clean} the desired data; and %as it is often cited that 20\% of a typical data source is incorrect or missing; and
(v)~{\it iterate} through these tasks using a workflow system, since data scientists often use these modules in different orders.


\dcv consists of two major components, as shown in 
Figure~\ref{fig:arch}.  
The {\it offline component} indexes and profiles data sets;  these profiles are stored  in a linkage graph, which is used to process workflow queries online. 
Data sets in \dcv consist of structured data, which may be stored in relational databases, spreadsheets, or other structured formats. In the remainder of the paper, we use the term {\it tables} or {\it data sets} to refer to these.
The {\it online component} involves executing a user-supplied workflow that consists of a mix of discovery, join path selection, and cleaning operations on data sets, all supported via interactions with the linkage graph. We elaborate on each module in the remainder of this section, using Merck as an example.


\begin{figure}[!t]
\includegraphics[width=3.6in]{arch3.pdf}
\caption{\dcv Architecture}
\label{fig:arch}
\end{figure}


\stitle{[Discovery.]} 
A data scientist at Merck has a hypothesis, for example, {\it the drug Ritalin causes brain cancer in rats weighing more than 300 grams}.
His first job is to identify relevant data sets, both inside and outside of Merck, that might contribute to testing this hypothesis. Inside the company alone, Merck has approximately 4,000 Oracle databases and countless other repositories. The discovery component in \dcv will assist the scientist in finding tables of interest from all the Merck tables.

Although discovery queries are run as a part of the online workflow, they are supported by profiles and indexes that are built during the offline processing.  Building these profiles requires looking at both the schema and values of every table. Consequently, we employ linear algorithms to make this pre-computation tractable. This module, together with the corresponding data structures, is discussed in Section~\ref{sec:discovery}.

\stitle{[Linkage Graph Computation.]} The semantic linkage between attributes of discovered tables is needed so that the user can assemble relational schemas and run \textsf{SQL} queries on the discovered data sets. All the linkages that can be found in linear time are built in advance during offline stage. All the other linkages, such as \pkfk relationships, will be built in background or on-demand during the online stage. Our current prototype identifies \pkfk relationships, using inclusion dependencies. As such, the process of building the linkage graph can be run dynamically and on-demand; we discuss its details in Section~\ref{sec:stitching}.



\stitle{[Polystore Query Processing and Curation.]}
Since organizations such as  Merck have a variety of massive-scale data storage systems, it is not feasible to move all data to a central data warehouse. Also, it is neither economically nor technically practical to perform data processing on all of the thousands of databases in advance. 
\dcv is built using a polystore architecture~\cite{DBLP:journals/sigmod/DugganESBHKMMMZ15} that federates query processing across disparate systems inside an enterprise. Our plan is to leverage the \texttt{BigDAWG} polystore system~\cite{DBLP:journals/pvldb/ElmoreDSBCGHHKK15} to pull data from multiple underlying storage engines to compute the final result (or the {\em view}) that satisfies the user's specifications.

%\stitle{[Curation Polystore]} 

%In fact, we are building on the \texttt{BigDAWG} polystore system~\cite{DBLP:journals/pvldb/ElmoreDSBCGHHKK15}. The polystore architecture can pull data out of multiple underlying storage engines as needed. 

Obviously, data cleaning, data transformation and entity consolidation must be integrated with querying the polystore and constructing the desired user view.  
This is an expensive process with the human effort required to validate cleaning decisions being the most important cost. 
Since there may be multiple views that ``solve'' a data scientist's query, each with different accuracy and human validation cost, \dcv must estimate the cost of curating the possible views,
%to reason about the feasibility of using each one,
given a scientist's time budget. 
The above aspects will be discussed in Section~\ref{sec:curating}.


%Obviously, we want to run stitching in the background in advance to deliver the best possible response time. As noted in Section~\ref{sec:stitching}, our current prototype leverages PK-FK relationships (inclusion dependencies).
%We also plan to explore other possible relationships in the futu%The merger of polystores and data curation steps is discussed in Section~\ref{sec:curating}.re such as ???.
%Effectively, the result of stitching is a ``view'' of multiple data sources that contains the composite data of interest to the scientist.


%\stitle{[Cleanliness Estimation]} 
%Indeed, \dcv is an iterative process, the human effort will be involved in various modules. The expensive process in constructing the view mentioned above is the human effort required to validate cleaning decisions. Since there may be multiple views that ``solve'' a data scientist's query, each with different accuracy and human validation cost, \dcv must estimate the cost of curating the possible views,
%%to reason about the feasibility of using each one,
%given a scientist's time budget. 

%Estimating the cleanliness of a view entails constructing a model for the dirtiness of the data in each source data set, which we discuss in Section~\ref{sec:curating}. Also discussed in Section~\ref{sec:curating} is where in a query plan we should allocate a cleaning budget.


%\stitle{[Optimizing Stitching]} It is highly inefficient and wasteful to  discard expensive-to-construct materialized views after their initial use by a data scientist.  Hence, we assume that they are generally retained for future use.  Moreover, future materialized views  may be based off previously constructed ones or on original data sources.  As a result, there may be several ways to construct a new view, with different  costs and accuracy. Therefore, the data stitching problem must be revisited to deal with this materialization cost/accuracy trade-off.  This is the subject of Sections~\ref{sec:enhancedstitching}.\mourad{Why not merge this section with Data Stitching?}

\stitle{[Updates.]}  
If a source data set is updated, these updates must be incrementally propagated through the data curation pipeline to update downstream materialized views. 
In some cases, the human effort involved may be daunting and the materialized view should be discarded rather than updated. 
In addition, if a scientist updates a view, we need to propagate changes to other derived views, as well as back upstream to data sources, if this is possible. Section~\ref{sec:updates} discusses these  issues.



\stitle{[Workflow.]} 
\dcv offers a workflow engine whereby data scientists can iterate over its components in whatever order they wish,  Moreover, they need to be able to undo previous workflow steps and perform alternate branching from the result of a workflow step.  Section~\ref{sec:workflow} discusses our workflow management ideas.

\smallskip

We describe the state of the current implementation of \dcv in Section~\ref{sec:wild}. We also report on initial user experience for two use cases: the MIT data warehouse and Merck. We conclude with final remarks and an outline of our future research plans in Section~\ref{sec:conclusion}.

\smallskip
%\stitle{Our contributions.}
The main contribution of \dcv is that it is an ``end-to-end'' system. In contrast, there has been much work on ``point solutions'' that solve small pieces of the overall problem. For example, Data Wrangler~\cite{2011-wrangler} and DataXFormer~\cite{DBLP:conf/icde/AbedjanMIOPS16} automate some aspects of cleaning and transforming data, Data Tamer~\cite{DBLP:conf/cidr/StonebrakerBIBCZPX13} mainly performs schema mappings and record linkage, and DeepDive~\cite{DBLP:journals/pvldb/ShinWWSZR15} extracts facts and structured information from large corpora of text, images and other unstructured sources. However, no solution performs  discovery, linkage graph computation, and polystore operations in concert. In addition, we have recently studied several representative data cleaning systems on a collection of real world data sets``from the wild''~\cite{DBLP:journals/pvldb/AbedjanCDFIOPST16} and we found out that there was no cleaning Esperanto. Hence multiple tools are required to achieve reasonable accuracy and point solutions neither offer enough functionalities nor achieve acceptable performance. Therefore an ensemble approach is required.


% !TEX root = ../main.tex
\section{Discovery}
\label{sec:discovery}

The goal of the data discovery module is to find relevant data among multiple,
perhaps millions of, datasets that spread across the storage systems of an
enterprise.

A naive solution is to ask an expert (if one is available), or to perform
manual exploration by inspecting datasets one by one, which is obviously
time-consuming and prone to missing relevant data sources.  Recently, we are
witnessing efforts to automate this process, such as
Goods~\cite{DBLP:conf/sigmod/HalevyKNOPRW16} or
InfoGather~\cite{DBLP:conf/sigmod/YakoutGCC12}. Unlike these systems
that are designed to answer a particular use case, our discovery module aims to
collect enough information about the datasets and to expose them to upper layers
of \dcv allowing the orchestration of query execution, which is the ultimate
goal.

The data discovery module narrows down the search for relevant data from
$n$ data sources to a handful number of $m$ data sources, upon which we can afford to compute finer-grained relationships.
\dcv discovery supports a user-level API to search for relevant data using a
variety of techniques such as schema search and similar content. 
%Discovery is extensible; an engineer can extend the supported relationships to cover other relevant use cases.

\subsection{Data Profiler}

We summarize each column of each table into a {\em profile}. Each profile contains a signature, which is a domain-dependent,
compact representation of the original content.  For numerical data, we use the
data distribution, and for textual data,  a vector of the most significant words
in the column.  Our profiles also contain information about data cardinality,
data type, and numerical ranges if applicable~\cite{profiling_survey}.

Note that finding pairwise relationships between each attribute profile is an $O(n^2)$ computation, which will not scale.  
We thus rely on locality sensitive hashing
(LSH)~\cite{DBLP:conf/compgeom/DatarIIM04} to address this issue. Specifically,
we hash each signature and group together those that collide.  These are more
likely to be similar; we only need to perform comparisons among the nodes in
each cluster, ignoring cross cluster comparisons.  This approach avoids $n^2$
operations and appears to perform well in practice.

%\subsection{Data Discovery Components}
%
%The data discovery module consists of two conceptually different components, a
%data profiler and a graph builder, that collaborate to build the linkage graph
%indicated in Figure~\ref{fig:arch}. 
%
%\begin{enumerate}
%\item {\bf Data profiler.} This component summarizes each column of each table
%into a profile. Each profile contains a signature, which is a domain-dependent,
%compact representation of the original content.  For numerical data, we use the
%data distribution, and for textual data,  a vector of the most significant words
%in the column.  Our profiles also contain information about data cardinality,
%data type, and numerical ranges if applicable~\cite{profiling_survey}.
%
%\item {\bf Graph builder.} Using the profiles, the graph builder finds
%relationships among nodes and represents them as edges. Examples of
%relationships are \emph{content similarity} based on measuring the similarity
%distance between signatures and \emph{schema similarity} that captures the
%similarity of the labels of the different nodes. Another kind of relationship
%captures the hierarchical relationship between a table and its columns.
%\nan{Each node is a column, or a table? The graph is not formally defined.}
%\sibo{The definition of node appears in Section \ref{introduction}, but it will
%be better to recap the definition again.}
%
%\end{enumerate}
%
%The graph resulting from this process can be used to answer discovery queries.
%Discovery queries vary from simple keyword search, useful to quickly glance over
%data, to more involved queries such as table augmentation, to add attributes of
%interest to known tables. We discuss the queries at the end of the section. 

\subsection{Data Profiling at Scale}

To run at scale, discovery relies on sampling to estimate the cardinality of
columns and the quantiles for numerical data. This avoids having to sort each
column, which is not a linear computation.  More generally, this avoids a
requirement to copy or update the data, reducing memory consumption.

The data profiler architecture is a pipeline of stages, with each stage
responsible for performing a different computation. For example, the first stage
performs basic denoising of the data, such as removing empty values and dealing
with potential formatting issues of the underlying files, e.g. figuring out the
separator for a CSV file.  The second stage determines the data type,
information that is propagated along with the data to the next stages. The rest
of the stages are responsible for computing cardinalities, performing quantile
estimation, e.g., to determine numerical ranges.

The profiler has connectors to read data from different data sources, such as
HDFS files, RDBMS or CSV files. The data is then streamed through the profiler
pipeline which outputs the results
%and  at the end the resulting attribute profile is sent
to a store from where it is later consumed by the graph builder.
%At the same time,
The profiler works in a distributed environment using a lightweight coordinator
that splits the work among multiple processors, helping to scale the
computation.

%The discovery module benefits from our distributed architecture that allows us to parallelize profiling as well as graph building, further speeding up execution times. Other kinds of relationships such as finding FK-PKs are too expensive to compute at scale for the whole graph.  These are left to the stitching module, which is described next.


\subsection{Discovery Queries}
\label{subsec:api}

The discovery system permits users to submit \emph{schema retrieval queries},
that are used to find schemas, i.e., datasets, relevant to the task at hand. For
example, a user may use discovery to find all datasets that contain names of
employees in a company with a schema retrieval query that search for all
attributes referring to employee or name, and then write a SQL query (data
retrieval query) to filter from a list. Next we give some examples of some functions
currently available to users:

\begin{itemize}
\item \textbf{Fill-in schema}. Given a set of names, the discovery system
retrieves and shows tables or groups of (probably joinable) tables that contain
the desired attributes. The core of the operation consists of finding tables
with similar attribute names to the provided ones.
\item \textbf{Extend attribute.} Similar to the extend operator of Octopus
\cite{octopus}, or the ABA operation of Infogather \cite{DBLP:conf/sigmod/YakoutGCC12}, discovery
can extend tables of interest with additional attributes requested by users. The
core of the operation consists of finding matches to the current table, and then
retrieving the attributes that do not appear in the original table.
\item \textbf{Subsumption relationships.} Given some table of reference,
discovery can provide a list of tables or groups of tables that have some form
of subsumption relationship with respect to the reference one. 
\item \textbf{Similarity and lookup.} Discovery can also be used to find schemas
that contain data similar to some provided schema (attribute or table), as well
as to find schemas that contain certain values, i.e., kewyord search for
textual data and range search for numerical data.
\end{itemize}

Discovery works as an information retrieval system: a schema retrieval query
does not have a specific answer, instead it returns a ranked set of results.
Hence, ranking appropiately the results from the hundreds of underlying data
sources falls in the domain of discovery, and it is of great importance to reduce
the amount of work that downstream components to discovery within Data Civilizer
must perform.


\section{Stitching}
\label{sec:stitching}

\notera{Dong}

% !TEX root = ../main.tex
\section{Polystore Query Processing}
\label{sec:curating}

\dcv uses the BigDAWG polystore~\cite{DBLP:journals/pvldb/ElmoreDSBCGHHKK15}. BigDAWG  consists of a middleware query optimizer and executor, and shims to various local storage systems.We assume that a user has run discovery and that the corresponding linkage graph has been computed. We can thus identify a collection of join paths that can materialize a composite table of interest to the user. Also, assume that the user has identified the subset of each source table in which he is interested. For any join path, it is now straightforward to construct the BigDAWG query that materializes the view specified by each join path. Next we discuss how to choose a join path from all the possible ones to materialize.
	

% in Section~\ref{subsec:selectview}. As one of the most important criteria to select the join path is its cleanliness, we then propose a model to estimate the cleanliness of the join path in Section~\ref{subsec:model}. When the join path is selected, we need to clean it with a given budget. We develop a method to integrate the cleaning operations in the query plan that materializes the view in Section~\ref{subsec:gain}.


\subsection{Selecting a View to Materialize}

The conventional data federation wisdom is to choose the join path that minimizes the query processing cost. However, this ignores data cleaning issues, to which we now turn. To achieve high quality results, one has to clean the data prior to querying the table. For example, if one has a data value, \code{New Yark}, and wants to transform it to one of its airport codes \code{(JFK, LGA)}, then one must correct the data to \code{New York}, prior to the airport code lookup. Obviously, cleaning usually entails a human specification of the corrected value or a review of the value produced by an automatic algorithm. Hence, it is expensive in human resources, which we believe is generally ``the high pole in the tent''. As such, one goal of \dcv is to choose the join path that produces the highest quality answer and not the one that is easiest to compute. 

In \dcv, a user has to decide how to trade off data quality and cleaning cost. 
\dcv defines two parameters under the user control.

\begin{enumerate}
\item Minimize cost for a specific cleanliness metric. In this case, the user requires the data to be a certain percentage, \emph{P}, correct and will spend whatever it takes to get to that point.

\item Maximize accuracy for a specific cost. In this case, the user is willing to spend \emph{M} and wishes to make the data as clean as possible.
\end{enumerate}

Sometimes the user is the one actually cleaning the data. In this case, (s)he can use \emph{P} and \emph{M} to quantify the value of her/his time. 
In other cases, cleaning is performed by other domain experts, who generally need to be paid. In this case, \emph{P} and \emph{M} are statements about budget priorities.

Sometimes the users may prefer the join path that yields largest view size. We can combine all the facets by a linear function as the criteria for join path selection. \dong{To cite the paper Ihab mentioned in the memting}

As a result, \dcv must make the following decisions.  First, it must make an assessment of the cleanliness of the result of any given join path.  We treat this issue in Section~\ref{subsec:model}. Then, we need to choose where to place, in the resulting query plan, data cleaning operations to be the most efficient. This is the topic of Section~\ref{subsec:gain}.


\subsection{The Cleanliness Model}\label{subsec:model}

In \dcv, we collect information about the errors in each data set. We first describe the two kinds of error information we collect. First, we need the accuracy metrics for each column, namely an estimation for the percentage of the column which is erroneous. Second, we need a ``disorder metric'' that indicates the distance between an incorrect value and its ground truth. For numeric values, we use the average and standard deviation as the disorder metric. For example, if salary errors average 5\% with standard deviation 2\%, then $\frac{2}{3}$ of the errors are less than 7\%. \footnote{\sibo{Where the 2/3 comes from? The distribution is not given in the context}}. For text values, we use the ``lexical distance'' (such as distance based on WordNet~\cite{WordNet,DBLP:journals/cacm/Miller95} and edit distance) as the disorder metric. For example, if an address field routinely confuses ``\textit{road}'' and ``\textit{street}'', but very rarely gets the name of the street wrong, then the lexical distance is very small. %\srm{We should explicitly define different metrics for lexical and numeric.  Mean +/- std dev doesn’t make sense for lexical, I think.}


Next we discuss how to collect the error information. There are multiple ways to achieve this and they can be composed. First, we can have the DBA who added the table make an estimate for the error information. Second, we can estimate the number of null values and outliers in the column as errors. Third, we can conduct a similarity join on the foreign key and primary key, take the mismatching values as errors, and use this error information for the other columns. Fourth, we can cluster the tables with similar primary key such that their records may have the same semantic, identify the same columns in the cluster, and use majority vote to estimate the cleanliness of each column. We can also have the DBA give the textual semantic description of the table as well as each column in the table to refine the table clustering and common column finding. Next we discuss propose a cleanliness model to estimate the cleanliness of a join path using the error information.


%To answer a query, \dcv generates a sub-graph for the query from the available join graph (which is the sub-graph of the linkage graph that only contains FK-PK edges) that covers all the joins in the query. For a given query, there may be multiple sub-graphs and we provide a multi-faceted interface for the user to pick one. One facet is the one with maximum cleanliness. Next we discuss how to achieve this according to the following cleanliness model.


We first define the cleanliness of an edge in the join graph. We first consider the join on textual values. Given two linked nodes \RX and \SY in the graph, when we align the values in \RX and \SY using the maximum matching defined in Section~\ref{subsec:eind}, we can tolerate some errors. Suppose that the maximum lexical distances in the two nodes are respectively \Dis(\RX) and \Dis(\SY). The distance between two values in \RX and \SY that represent the same object should be within $\Dis(\RX)+\Dis(\SY)$. Thus we limit the distance between the pair of aligned values in the maximum matching by $\Dis(\RX)+\Dis(\SY)$. Then the fraction of the values in the foreign key that appear in the maximal matching is the cleanliness of the edge. Similarly, we can do the same thing on numeric values with numeric metric.

We define the cleanliness of a node \RX simply as its accuracy \Acc(\RX), which means the percentage of values in \RX that are not erroneous.

Given a join path, its cleanliness is obtained by multiplying the cleanliness values of all the edges, and the nodes involved in predicates which is necessary since errors can propagate along the edge and is dominated by the dirtiest one.



\subsection{Query Plan with Data Cleaning Operation}
\label{subsec:gain}

Obviously expensive cleaning should be performed on as few records as possible. Hence, we would like to insert cleaning operations as late in the query plan as possible. 
Unfortunately, if we run a query with the following predicate:

\vspace{.5em}
\dots \textsf{where} $name = $ ``\code{New York}''
\vspace{.5em}

\noindent Then the misspelled city name,  \ie \code{New Yark}, will not be found, and accuracy will suffer. One solution is to clean the entire source data sets to avoid such errors, an expensive proposition indeed. If the penalty is small, then we can insert cleaning ``late'' in the query plan. On the other hand, if the penalty is large, then we must insert cleaning earlier, even though at much higher cost. We formalize our idea as follows.


For each edge in the join path, if we put the cleaning operation ahead of the join operation, i.e., we only clean the tuple pairs in our maximal matching, we spend cleaning cost linear to the number of foreign keys and can only expect to achieve the same accuracy  as the cleanliness of the edge. In contrast, if we first clean all the possible tuple pairs of the foreign key and primary key and then generate the real join results, we can achieve 100\% accuracy with a cleaning cost quadratic to the number of foreign keys. Similarly, if we put the cleaning operation behind the where clause, we can only expect to achieve the same accuracy of the column with the cost proportional to the number of tuples satisfying the where clause. Otherwise, we can achieve 100\% accuracy with the cost proportional to the number tuples in the whole column.


In this setting, we can use dynamic programming to obtain a query plan with cleaning operations that achieve the maximum accuracy gain with limited cleaning cost budget or achieve the desired accuracy with the smallest cleaning cost budget. More specifically, for each node and edge in the join graph, we can choose not to clean it, clean it before the query operation (i.e., join or selection), or clean after the query operation. Each with different cleaning cost. With dynamic programming, we can achieve an optimal strategy.



%% !TEX root = ../main.tex
\section{Cleanliness Estimation}
\label{sec:cleanliness}

After running the  discovery and stitching, a user identifies a virtual relation (or view) to query.  This includes identifying the subset of each source table of interest.  In other words, the user has defined the join path using stitching and the predicates that subset the tables in some other way.   Directly querying this view or the base tables without any cleaning entails two risks: (i) Missing results (false positives) and (ii) Reporting wrong results (false negatives). For example,  if one has a data value, ?New Yark? in a column city, and wants to perform a join on city, then ?New Yark? should be corrected to ?New York?, otherwise the join will miss some results (False negatives). Now if the column contains wrong but valid values, then these will ?wrongly? be joined. 

In addition to the user query, we assume the following input are given, namely sizes of base relations, cost of cleaning a cell (could be per relation, per column, and per tuple or we could assume one unit cost for any cell),  accuracy metrics per column which gives an estimate for the percentage of the column which is erroneous,  user?s goal for cleanliness P in terms of percentage of erroneous data,  and cleaning budget M which is the budget the user is ready to spend on cleaning.

Since cleaning usually entails human efforts, we need a (i) way to estimate the cleanliness of the query results and (ii) how is it affected by the type of operations and (iii) how it can be changed by applying cleaning (which entails human efforts) on specific data values. Assuming SPJA (selection, projection, join, aggregation, and grouping) queries, each of the operation will have a different effect on the cleanliness of the results. More specifically,  with selection, the ?amount? of errors will go down, with projection,  the amount of errors will go down, error may even disappear if projection drops erroneous columns and keep clean ones, with join, the amount of errors may increase, and with grouping and aggregate, the amount of errors may decrease. 

\mourad{It will be good if we can devise some sort of a formula for estimating cleanliness. Maybe we should also move the text on ?disorder metric? here.}
%\section{Enhanced Data Stitching}
\label{sec:enhancedstitching}

% !TEX root = ../main.tex
\section{Updates}
\label{sec:updates}

Real-world data is rarely static, which is exactly the scenario that Data Civilizer faces in Merck and the M.I.T. Data Warehouse.

We categorize three types of updates managed by Data Civilizer.

\stitle{(1) Insertions/deletions on source tables.} 
This happens when there is a change of the table, e.g., insertions of new procurement records in the M.I.T. Data Warehouse. This may also happen when some data sources get cleaned or transformed on demand (see Section~\ref{sec:curating}).

\stitle{(2) Replacement of source tables.} 
Large companies typically rely on both internal and external information to build their knowledge. For instance, Merck will collect published standard medical names from the World Health Organization (WHO) to help construct their own ontology. These information will be updated periodically by WHO. Sometimes, even the format will be changed, e.g., from a JSON file to a CSV file.

\stitle{(3) Updating MVs.} 
MVs might be created in cascade, and the human effort for data curation might happen in any layer. 

In response to the above three types of updates, Data Civilizer uses three strategies correspondingly.

\stitle{(i) MV maintenance.}
In the simplest case of small changes over the source data, such as the case (1) above, Data Civilizer will leverage the mature techniques for maintaining materialized views (see~\cite{DBLP:journals/debu/GuptaM95} for a survey), which has been widely deployed in many commercial DBMSs. In such a way, Data Civilizer incrementally propagates the updates through the data curation pipeline to update downstream MVs.  

\stitle{(ii) Provenance management.}
In some cases such as the above case (2), the human effort involved may be daunting for updating the MVs. In these scenarios, the MVs should be discarded rather than updated. Naturally, there is need for a component that natively supports the versioning or branching of data to enable concurrent analysis, cleaning, integration, or curation of data across data sources. 
%
Data Civilizer leverages Decibel~\cite{DBLP:journals/pvldb/MaddoxGEMPD16}, a system developed by MIT, for this purpose.

\stitle{(iii) Descriptive and prescriptive data cleaning.}
Sometimes, a scientist may curate directly his MV such as the above case (3), which triggers some updates that must be propagated to other descendent data sets, as well as back upstream to data sources. To perform this, we leverage the technique in~\cite{DBLP:conf/sigmod/ChalamallaIOP14}, a system developed by QCRI and Waterloo. In the MV, the updates will be captured based on human data curation, which will be transformed at the source level to prescribe actions to solve them. 

% !TEX root = ../main.tex
\section{Tractable Curation Workflow}
\label{sec:workflow}

The process of data curation on demand might entail various iterations of data discovery, data stitching, and data curation steps. Especially the data curation requirement might consist of various cleaning and transformation procedures that have to be guided by a user. To facilitate the user in this is process, we exploit the fact that the type of data curation and preparation needed inside a company might follow repretitive patterns. 
Therefore, 
we will store the sequence of data curation operations in a workflow registry together with the meta-data and signatures provided by the data discovery component for comparison with future data discovery scenarios. 

During later data discovery tasks, the signatures will be compared to those stored in the workflow registry and accordingly the related curation and preparation procedures will be proposed to the user. Furthermore, the workflow registry will be able to identify redundant and subsuming curation steps by profiling the input and output of each operation.
% !TEX root = ../main.tex
\section{Data Civilizer in the Wild}
\label{sec:wild}

We have deployed a preliminary prototype of Data Civilizer in two different
organizations. The MIT Datawarehouse is a group inside MIT responsible for
building and maintaining a data warehouse that integrates data from multiple
source systems. Merck, a big pharmaceutical company, manages large volumes of
data, which are managed by different storage systems. Some characteristics of
both organizations' data are shown in Table~\ref{tab:dataCivInTheWild}. Next we
provide detail on each organizations' requirements, and how we are using Data
Civilizer to help them.

\begin{table}
\caption{Deployment environments of Data Civilizer\notera{I'd remove use cases
from here}}\label{tab:dataCivInTheWild}
\begin{tabular}{|l|l|l|l|}
\hline
Organization & \# databases & \# tables & Use cases\\
MIT DWH & & & \\
Merck & & & \\
%Doha Traffic Signals & 9 & millions & ??? \\
\hline
\end{tabular}
\end{table}

\subsection{MIT Data warehouse}

The MIT Datawarehouse is a team within MIT that manages a warehouse that
integrates data from hundreds of source databases from around the campus. One of
the key tasks of the team is to assist its customers---any personnel from within
MIT---in answering questions they have, for example, to create reports. The
warehouse contains around 1TB of data spread across 3K tables aproximately.

A typical customer of the warehouse will present a question, for whith the
members will need to find relevant tables manually. They create a view that is
accessed by the customer to solve the question at hand. We describe next some of
the common use cases we have found:

\mypar{Fill in virtual schema} When a customer arrives with a question such as:
\emph{I need to create a report with the \textbf{gender distribution of the faculty per
department and year}}, the data warehouse personnel can use Data Civilizer to
find all the tables that contain schema names similar to the attributes exposed
by the query, \eg gender, faculty name, department, year. 

\mypar{Table redundancy} Multiple views are created for different customers.
Many of them contain typically very similar data, as multiple customers are
interested in similar items. To reduce the redundancy of data, Data Civilizer
helps to detect complementary as well as repeated sources. This sheds light on
the status of the warehouse and helps to maintain it tidy and minimal.

We deployed Data Civilizer as a \ldots service and allowed the users to interact
with the system through the graphical interface. Data Civilizer was able to
\ldots

\subsection{Merck}

Merck is a big pharmaceutical company that manages large volumes of data spread
across around 4K databases, plus several data lakes. The use cases are varied,
however there is a common case. One of the data assets of any pharmaceutical
company are internal databases of chemical compounds and structures. Usually,
these are more valuable when integrated with external, well-known and curated
databases, such as PubChem \cite{pubchem}, Chembl \cite{chembl} or Drugbank
\cite{drugbank}. We describe two common use cases that occur in this context:

\mypar{Identify entities} One single chemical entity may be referred to with
different identifier format in different databases. Chemical identifiers have
been a subject of research in the bioinformatics community: multiple different
formats have been proposed with different properties desirable according to the
scenario. Data Civilizer helps by mapping the multiple representations of the
identifiers, therefore facilitating the identification of entities across
multiple databases, public and internal.

\mypar{Enrich data} One of the reasons for the existence of multiple chemical
databases is that each puts an emphasis on different information. Analysts
typically face situations in which they are interested in a set of attributes
that are spread across different tables on different databases. Data Civilizer
helps to detect such attributes and bring them together \emph{on-demand} to
serve the users' purpose. 

% !TEX root = ../main.tex
\section{Conclusion and Future Work}
\label{sec:conclusion}

In this paper, we propose an end-to-end big data management system named \dcv with the goal of discovering relevant data to the user specific tasks, link the relevant data for better usage, cleaning the data with limit budget, handle data updates, and enable iteratively processing the data. We build a module for each of the requirement above. As data in large companies are usually across multiple data storage platforms, we build our system in a polystore architecture. Moreover, we propose to place the data cleaning operation in the query plan, and trade-off the query result quality with the cleaning costs. We deployed our preliminary system on two different institutions, MIT and Merck and got positive feedbacks from the users.


\dong{Future work goes here.}


\bibliographystyle{abbrv}
\balance%\footnotesize
\bibliography{main}
\vskip 1em

\end{document}

% End

%%% Local Variables:
%%% mode: latex
%%% TeX-master: t
%%% End:
