\documentclass{sig-alternate-05-2015}

\usepackage{graphicx}
\usepackage{balance}
\usepackage{caption}
\usepackage{hyperref}
\usepackage{url}
\usepackage{microtype}
\usepackage{paralist}
\usepackage{booktabs}
\usepackage{amssymb}
\usepackage{amsmath}
\usepackage{mathtools}
\usepackage{listings}
\usepackage{cite}
\usepackage{subscript}
\usepackage{enumitem}
\usepackage{xspace}
\usepackage{amssymb}

\let\proof\relax
\let\endproof\relax

\usepackage{amsthm}
\usepackage{float}
\usepackage[algoruled, linesnumbered, vlined]{algorithm2e}
\usepackage[utf8]{inputenc}

\usepackage{inconsolata}

\usepackage{prp-macros}

\newcommand*\samethanks[1][\value{footnote}]{\footnotemark[#1]}
\newtheorem*{example*}{Example}

\lstset{ %
   belowskip=-2em,
   backgroundcolor=\color{white},   % choose the background color; you must
   basicstyle=\scriptsize\ttfamily,        % the size of the fonts that are used for the code
   breakatwhitespace=false,         % sets if automatic breaks should only happen at whitespace
   breaklines=true,                 % sets automatic line breaking
   captionpos=b,                    % sets the caption-position to bottom
   commentstyle=\color{Brown},    % comment style
   deletekeywords={...},            % if you want to delete keywords from the given language
   escapeinside={\%*}{*)},          % if you want to add LaTeX within your code
   extendedchars=true,              % lets you use non-ASCII characters; for 8-bits encodings only, does not work with UTF-8
   frame=,                    % adds a frame around the code
   keepspaces=true,                 % keeps spaces in text, useful for keeping indentation of code (possibly needs columns=flexible)
   keywordstyle=\color{MidnightBlue},       % keyword style
   language=,                 % the language of the code
   morekeywords={*,range,var},            % if you want to add more keywords to the set
   numbers=none,                    % where to put the line-numbers; possible values are (none, left, right)
   numbersep=5pt,                   % how far the line-numbers are from the code
   numberstyle=\tiny\color{gray}, % the style that is used for the line-numbers
   rulecolor=\color{black},         % if not set, the frame-color may be changed on line-breaks within not-black text (e.g. comments (green here))
   showspaces=false,                % show spaces everywhere adding particular underscores; it overrides 'showstringspaces'
   showstringspaces=false,          % underline spaces within strings only
   showtabs=false,                  % show tabs within strings adding particular underscores
   stepnumber=1,                    % the step between two line-numbers. If it's 1, each line will be numbered
   stringstyle=\color{RedOrange},     % string literal style
   tabsize=2,                       % sets default tabsize to 2 spaces
   title=\lstname,                   % show the filename of files included with \lstinputlisting; also try caption instead of title
   moredelim=[is][\color{blue}\bfseries\underbar]{@}{@},
}

\lstdefinestyle{myCSharp}{
   language={[Sharp]C},
   basicstyle=\scriptsize,
   belowskip=-2em,
   backgroundcolor=\color{white},   % choose the background color; you must
   basicstyle=\scriptsize\ttfamily,        % the size of the fonts that are used for the code
   breakatwhitespace=false,         % sets if automatic breaks should only happen at whitespace
   breaklines=true,                 % sets automatic line breaking
   captionpos=b,                    % sets the caption-position to bottom
   commentstyle=\color{Brown},    % comment style
   deletekeywords={...},            % if you want to delete keywords from the given language
   escapeinside={\%*}{*)},          % if you want to add LaTeX within your code
   extendedchars=true,              % lets you use non-ASCII characters; for 8-bits encodings only, does not work with UTF-8
   frame=,                    % adds a frame around the code
   keepspaces=true,                 % keeps spaces in text, useful for keeping indentation of code (possibly needs columns=flexible)
   keywordstyle=\color{MidnightBlue},       % keyword style
   morekeywords={var},            % if you want to add more keywords to the set
   numbers=none,                    % where to put the line-numbers; possible values are (none, left, right)
   numbersep=5pt,                   % how far the line-numbers are from the code
   numberstyle=\tiny\color{gray}, % the style that is used for the line-numbers
   rulecolor=\color{black},         % if not set, the frame-color may be changed on line-breaks within not-black text (e.g. comments (green here))
   showspaces=false,                % show spaces everywhere adding particular underscores; it overrides 'showstringspaces'
   showstringspaces=false,          % underline spaces within strings only
   showtabs=false,                  % show tabs within strings adding particular underscores
   stepnumber=1,                    % the step between two line-numbers. If it's 1, each line will be numbered
   stringstyle=\color{RedOrange},     % string literal style
   tabsize=2,                       % sets default tabsize to 2 spaces
   title=\lstname,                   % show the filename of files included with \lstinputlisting; also try caption instead of title
   moredelim=[is][\color{blue}\bfseries\underbar]{@}{@},
}

\lstdefinestyle{mySQL}{
   language={SQL},
   basicstyle=\scriptsize,
   belowskip=-2em,
   backgroundcolor=\color{white},   % choose the background color; you must
   basicstyle=\scriptsize\ttfamily,        % the size of the fonts that are used for the code
   breakatwhitespace=false,         % sets if automatic breaks should only happen at whitespace
   breaklines=true,                 % sets automatic line breaking
   captionpos=b,                    % sets the caption-position to bottom
   commentstyle=\color{Brown},    % comment style
   deletekeywords={...},            % if you want to delete keywords from the given language
   escapeinside={\%*}{*)},          % if you want to add LaTeX within your code
   extendedchars=true,              % lets you use non-ASCII characters; for 8-bits encodings only, does not work with UTF-8
   frame=,                    % adds a frame around the code
   keepspaces=true,                 % keeps spaces in text, useful for keeping indentation of code (possibly needs columns=flexible)
   keywordstyle=\color{MidnightBlue},       % keyword style
   morekeywords={var},            % if you want to add more keywords to the set
   numbers=none,                    % where to put the line-numbers; possible values are (none, left, right)
   numbersep=5pt,                   % how far the line-numbers are from the code
   numberstyle=\tiny\color{gray}, % the style that is used for the line-numbers
   rulecolor=\color{black},         % if not set, the frame-color may be changed on line-breaks within not-black text (e.g. comments (green here))
   showspaces=false,                % show spaces everywhere adding particular underscores; it overrides 'showstringspaces'
   showstringspaces=false,          % underline spaces within strings only
   showtabs=false,                  % show tabs within strings adding particular underscores
   stepnumber=1,                    % the step between two line-numbers. If it's 1, each line will be numbered
   stringstyle=\color{RedOrange},     % string literal style
   tabsize=2,                       % sets default tabsize to 2 spaces
   title=\lstname,                   % show the filename of files included with \lstinputlisting; also try caption instead of title
   moredelim=[is][\color{blue}\bfseries\underbar]{@}{@},
}


\graphicspath{{images/}}

\renewcommand{\labelitemi}{$\circ$}
\renewcommand{\labelitemii}{$-$}

\captionsetup[figure]{singlelinecheck=false, margin=0pt, font={sf,small},
  labelfont=bf, justification=centering}
\captionsetup[table]{singlelinecheck=false, margin=0pt, font={sf,small},
  labelfont=bf, justification=centering}
\captionsetup[lstlisting]{singlelinecheck=false, margin=0pt, font={sf,small},
  labelfont=bf, justification=centering}

\setlength{\textfloatsep}{5pt}
\setlength{\dbltextfloatsep}{5pt}
\setlength{\abovecaptionskip}{5pt}
\setlength{\floatsep}{5pt}


\renewcommand\AlCapFnt{\small\sffamily}

\def\Snospace~{\S{}}
\renewcommand*\sectionautorefname{\Snospace}
\renewcommand*\subsectionautorefname{\Snospace}
\renewcommand*\subsubsectionautorefname{\Snospace}
\newcommand{\corollaryautorefname}{Corollary}
\newcommand{\propositionautorefname}{Prop.}
\newcommand{\definitionautorefname}{Def.}
\renewcommand{\subsectionautorefname}{\Snospace}
\renewcommand{\figureautorefname}{Fig.}
\renewcommand{\equationautorefname}{Eq.}
\newcommand{\exampleautorefname}{Example}
\newcommand{\problemautorefname}{Problem}
\renewcommand{\algorithmautorefname}{Alg.}

\DeclareMathOperator*{\argmax}{arg\,max}
\DeclareMathOperator*{\argmin}{arg\,min}

\newcommand\mycommfont[1]{\footnotesize\rmfamily\textcolor{black}{#1}}
\SetCommentSty{mycommfont}

\newcommand{\myparnospace}[1]{\noindent \textbf{#1}}

\newcommand{\sys}{\textsc{Saber}\xspace}
\newcommand{\saber}{\sys}
\newcommand{\Saber}{\sys}
\newcommand{\qtsize}{$\varphi$\xspace}

\newcommand{\Sbs}{Stream batch size\xspace}
\newcommand{\sbs}{stream batch size\xspace}

\newcommand{\notemw}[1]{\textcolor{orange}{{\bf MW: {#1}}}}
\newcommand{\notealw}[1]{\note{\color{BlueGreen}[\textsc{ALW:} #1]}}

\newcommand{\CM}[1]{\textsf{CM\textsubscript{#1}}\xspace}
\newcommand{\SG}[1]{\textsf{SG\textsubscript{#1}}\xspace}
\newcommand{\LRB}[1]{\textsf{LRB\textsubscript{#1}}\xspace}
\newcommand{\zia}[1]{\hl{\footnote{\hl{Zia: #1}}}}
\newcommand{\dong}[1]{\textcolor{red}{\bf Dong: {#1}}}
\newcommand{\eat}[1]{}

\renewcommand{\ldots}{\ifmmode\mathinner{\ldotp\kern-0.1em\ldotp\kern-0.1em\ldotp}\else.\kern-0.13em.\kern-0.13em.\fi}

\renewcommand*{\UrlFont}{\ttfamily\smaller\relax}

\begin{document}

% ****************** TITLE ****************************************

\title{The Data Civilizer System}

% ****************** AUTHORS **************************************

\numberofauthors{1}


\author{
 \alignauthor
}


\maketitle

% !TEX root = ../main.tex

\begin{abstract}
In large companies, it is usually hard for users to find relevant data for their specific tasks, from simple keyword queries to more complex structured queries, since the data is often scattered, inconsistent, and updated frequently. In order to decrease the ``grunt work'' needed to facilitate the analysis of data in the wild, we present an end-to-end data management system, named \dcv. 
%
It has a {\em data discovery} module to identify the data that is relevant to user tasks.
It also contains a {\em linkage graph computation} module that links all discovered relevant data on demand.
In addition, its {\em query processing} module is built on top of a polystore architecture to compute a result for a given user task, which also integrates data cleaning operations.
In practice, different tasks might invoke the above modules in different orders, and might be iterative.
To cope with this, we implement a {\em workflow} engine to enable the arbitrary composition of different modules, as well as handling data updates.
We have deployed our preliminary \dcv system in two institutions, MIT and Merck. The users gave us positive feedback and indicated that the system indeed shortened their time and effort to prepare and analyze the data.
%In this paper, we present an end-to-end data management system, named \dcv, whose main purpose is to decrease the  ``grunt work''\mourad{I prefer http://dictionary.cambridge.org/dictionary/english/grunt-work.} needed to facilitate the 
%``mung work''  for  analyzing analysis of data in the wild. 
% To solve these problems, \dcv has a data discovery module to help users identify their interesting data. 
%We also develop a data stitching module to link all the relevant data for a better usage. 
%We build the system on top of a polystore architecture and integrate the data cleaning operations in the query plan. 
%In addition, as the data preprocessing conducted by the users are often repetitive, we design a workflow engine to enable the composition and the iteration over the above modules as well the handling of updates. We deployed our preliminary \dcv system on two institutions, MIT and Merck. The users gave us positive feedbacks and indicated that the system indeed shortened their time and effort to prepare and analyze the data.
\end{abstract}
\section{Introduction}
\label{introduction}

There is overwhelming anecdotal evidence that data scientists spend at least 80%
of their time finding, preparing, integrating and cleaning data sets which they
wish to analyze. The remaining 20\% of their time is spend doing the analysis
tasks, that comprise their job description.  One data officer (Mark Schrieber of
Merck) estimates the number is 98\%, in which case data scientists spend less
than one hour a week on tasks in their job description.

n this paper we present the architecture of Data Civilizer, a system under
construction at MIT, QCRI and Waterloo, whose purpose is to lower the “mung
work” faced by data scientists.  In the rest of this section we present the
environment at Merck to motivate the components of Data Civilizer.  We also
indicate an architecture diagram of our system.  Then, Section 2 – 6 present the
components of our system, followed in Section 7 by experiences “in the wild” at
both Merck and the M.I.T. Data Warehouse.  We expect to present a demo at CIDR
of our system running in the MIT environment.   Finally, Section 8 draws
conclusions and suggests next steps.

\subsection{Merck Environment}

Merck is a large decentralized drug company with about XXX employees, of which
YYY are data scientists.  An exemplar data scientist would come up with a
hypothesis, for example the drug ritalin causes brain cancer in rats weighing
more than 1Kg.  His first job is identify data sets, both within the Merck
firewall and outside that might contribute to resolving this question.  Inside
the firewall, Merck has some 4000 Oracle databases and countless other
repositories in which relevant data might reside.  Data Civilizer has a
Discovery component, discussed in Section 2,which assists the scientist with
finding data sets of interest.

Data sets identified by the Discovery module are invariably linked together by
intermediary data sets.  Hence, the next step is to construct  “data stitching”
paths among all of the data sets indentified during discovery.  This task is the
job of the Data Stitcher, which is discussed in Section 3. One can think of the
output of the data stitcher as one or more views on the underlying data. It is
now necessary to perform data curation on these multiple views.  This entails
extracting data from source data storage systems, performing schema integration
on the multiple views, transforming data into a common representation, cleaning
erroneous values from the source data sets, and performing entity consolidation
on resulting records.  Our Data Tamer system [ref] dealt with schema integration
and entity consolidation.  More recently, we wrote a system DataXformer [ref]
which supported transformations, and we examined a collection of data cleaning
systems [ref].  

Since Merck has a variety of data storage systems and exascale data volumes, it
is simply not reasonable to move all data to a central “data lake”.  Also, it is
not reasonable to perform data curation up front on enterprise data, as was
advocated by Data Tamer.  Instead Data Civilizer must be a pull-based system
that does data curation on demand, as data scientists need to access data to get
their work done.  Therefore, Data Civilizer must be based on a polystore
architecture [ref], that can pull data out of multiple underlying storage
engines on demand.  Obviously, data cleaning, data transformation and entity
consolidation must be integrated with querying the polystore.  In this way, a
key technical optimization is to push filters and joins through cleaning
operations and into the underlying data storage system wherever possible.  The
merger of polystores and data curation steps is discussed in Section 4, and we
term the resulting system a Curating Polystore.  

Optimizing such a curating polystore is the subject of the next two sections.
Materializing views is very expensive, because of the human effort involved,
when automatic algorithms are unsure of what to do.  Therefore, we must estimate
the cost of constructing the MV, because a scientist may not have the budget to
pay for the human effort involved.  This is the topic of Section 5.  It entails
constructing a model for how dirty the data in the source data sets actually is.
An ancillary topic is to estimate the cleanliness that can be achieved for a
given budget for cleaning activities.  In this way, a scientist can decide
whether or not he wishes to proceed with the project at hand.

Moreover, it is silly to discard expensive-to-construct materialized views (MV)
after their initial use by a data scientist.  Hence, we assume that they are
generally retained for future use.  Moreover, future MVs may be based off
previously constructed ones or on original data sources.  As a result, there may
be several ways to construct a new MV, with differing costs and available
accuracy (since each existing MV has some given accuracy, as noted above.
Therefore, the data stitching problem must be revisited to deal with this
materializiation cost/accuracy tradeoff.  This is the subject of Sections 6.

Section 7 than turns to update issues.   If a source data set is updated, then
updates must be incrementally propagated through the data curation pipeline to
update downstream MVs.  In some cases, the human effort involved may be
daunting, and the MV should be discarded rather than updated.  Lastly, if a
scientist updates his MV, we must propagate changes to other descendent data
sets, as well as back upstream to data sources. 

Section 8 then turns to our workflow system, whereby a data scientist can
iterate over our components in whatever order he wishes, undo previous workflow
steps, and perform alternate branching from given MVs. 

We then turn in Section 9 to the current implementation status of Data Civilizer
and indicate initial user experience at the MIT data warehouse, as well as
Merck.  Section 10 gives conclusions, and outlines our future research plans.


% !TEX root = ../main.tex
\section{Discovery}
\label{sec:discovery}

The goal of the data discovery module is to find relevant data among the
millions of datasets spread across many storage systems of modern organizations.
Suppose an analyst wants to answer the question: what is the monthly sales trend
by department? The analyst knows conceptually what data is needed to answer this
question (a table of sales, a table of departments, a table of products sold by
each department), but not which specific data sources (which relations in a
schema or what files in an HDFS deployment) contain such data. The typical
solution is to 1) ask an expert (if such a person is available), or to perform
manual exploration, inspecting datasets one by one (which is time-consuming and
prone to missing relevant data sources). We say this analyst is facing a data
discovery challenge.

The data discovery module narrows down the search of relevant data from the
thousands and millions of data sources to a handful of them that can then be fed
to stitching---that performs the final preparation before processing. Discovery
exports an API that can be used by users directly to search for relevant data,
e.g., schema search, similar content, etc, as well as by other modules in Data
Civilizer, such as stitching, to retrieve additional data that can be used, for
example, to enrich a table. Next we describe the general steps of data
discovery:

\subsection{Data Discovery Components}

The data discovery module consists of two conceptually different components that
collaborate with each other to build a skeleton of the linkage graph shown in
Figure~\ref{fig:arch}. The linkage graph is defined as $G=(V,E)$, where $E$ is a
multiset, as we permit different types of edges, \ie relationships, betweeen
$V$. To build this multigraph, discovery consists of two modules, the first one,
the \code{data profiler} is in charge of finding the $V$ in the graph, and the
second one, the \code{graph builder} is reponsible for finding $E$.

\begin{enumerate}
\item {\bf Data profiler.} We first accumulate knowledge of the data sources by
summarizing them into concise profiles. We can choose the granularity of those
profilers. In practice, we have found that building a profile per attribute is
sufficient for most use cases. A profile contains a signature, which is a
domain-dependent, compact representation of the original content. One example of
signature for numerical data is its distribution, and for textual data a vector
with the most significative terms. A profile also contains information about the
data cardinality, data type, and numerical ranges when it
applies~\cite{profiling_survey}.  

\item {\bf Graph builder.} The data profiler generates a set of $V$, and the
graph builder reponsibility is to find relationships among these, and represent
them as $E$, building the skeleton of the linkage graph. These relationships
help to navigate through the different sources ($V$). Examples of relationships
that the graph builder finds are \emph{content similarity} based on measuring
the similarity distance among the signatures that represent each $v$. Another
relationship is \emph{schema similarity} that captures the similarity of the
names of the different $v$. Other kind of relationships capture the hierarchical
relationship among $v$, for example, all attributes of a same relation are
connected, permitting quick navigation. Creating these relationships require in
general a pairwise comparison that would render the module too slow. We discuss
next some techniques we use to tame the scale.
\end{enumerate}

Once the linkage graph skeleton has been built, it can be accessed by a set of
data discovery primitives to explore and find relevant data. These primitives
are used by users and by other modules, such as stitching. In particular,
stitching also has write access to the linkage graph, that uses to add a new $e$
to the multiset of relationships, the expensive-to-compute FK-PK relationships.
Data discovery primitives can be composed into more complex data discovery
functions, and all of these combine through combinator operators to build
expressive discovery queries.  

\subsection{Data Profiler}

The profiling and summarization module is responsible for computing signatures
for each attribute in the dataset. A signature for finding similar attributes
must: (i) represent the attribute values in a compact way; and (ii) be
compatible with a similarity metric. 

Currently we have primitives for numerical and textual values. For numerical
values the module learns the probability distribution of the data. We use
non-parametric methods for this, as it is in general impossible to tune the
learning process for every attribute in the dataset due to the scale. We use
kernel density estimators for numerical values.  For textual values, we first
represent each attribute as a bag of words and then compute its TF-IDF vector.

\textbf{Data quality estimation.} As part of the profiling process and aside
from the signature extraction, the component is also responsible for detecting
data type, cardinality and performing some basic denoising to avoid common data
errors. All this information serves as a data quality estimation for other
modules in Data Civilizer, such as stitching and the query processor \ref{sec:4,
sec:5}). 

For efficient cardinality and quantile estimation, required to summarize
numerical data, we use sketches extensively. This allows for memory-efficient
computation. By treating the input data as a read-once stream we also avoid
throttling the original data sources and in general, allows higher performance
summary computation.

%\sibo{The details for entity extracting / sematic type analysis are missing.}
%
%To extract entity / analyze semantic type for textual data, we use {\em named
%entity recognizers (NER)} \cite{XXX} to detect typical semantic types, e.g.,
%person, date, and location. However, the cost to invoke the NERs is extremely
%high, and it is unfavorable to apply NERs to every value in the column. As an
%alternative, we adopt an early termination approach that stops the NER checking
%as soon as the module is confident about the semantic type, e.g., consistently
%identifying the same semantic type with new fed data in the column. This allows
%much higher performance without significantly scarifying the entity extraction /
%semantic type analysis accuracy.

%\section{Cleanliness Estimation}
\label{sec:cleanliness}


\subsection{Graph Builder} The input of the graph builder is a set of profiles,
that are represented as nodes in a multigraph. The goal of the builder component
is to find content and schema similarity relationships among these nodes, which
are then represented as edges. In this way, the multigraph will have an edge of
type \emph{content-similarity} between nodes A and B if these are similar, and
one of type \emph{schema-similarity} if the schema names are similar too. Each
edge is scored with a weight that represents the strength of the relationship:
the particular meaning depends on the edge semantics.

Although this approach creates a logically complete graph, in practice it is
possible to define a minimum similarity threshold to prune edges. Building the
graph requires a pair-wise operation among all attributes in the
entire dataset, a $N^2$ operation. To scale this process, we are introducing
approximate nearest neighbor techniques based on locality sensitive hashing \cite{DBLP:conf/compgeom/DatarIIM04}
that helps to scale the process. In addition, we have built a distributed
prototype that can ingest data and compute signatures and search paths in
parallel.



% !TEX root = ../main.tex
\section{Linkage Graph Computation}
\label{sec:stitching}

In this section, we discuss how to build the linkage graph, as shown in Figure~\ref{fig:arch}. We first show how to build the data profiles in linear time which can be fed into the linkage graph and then formally define our graph model in Section~\ref{subsec:graphbuild}. Finally we discuss a specific kind of linkage in the graph in Sections~\ref{subsec:eind} and~\ref{subsec:refine}.



\subsection{Data Profiling at Scale}\label{subsec:profile}

The key idea is to summarize each column of each table into a {\em profile}.
A profile consists of one or more {\it signatures};  each signature summarizes
the original contents of the column into a domain-dependent, compact
representation of the original data.  By default, signatures for numerical values consist of a
histogram representing the data distribution and; for textual values they consist of a vector
of the most significant words in the column.  Our profiles also
contain information about data cardinality, data type, and numerical ranges if
applicable~\cite{profiling_survey}.

%Note that finding pairwise relationships between each attribute profile is an
%$O(n^2)$ computation, which will not scale.  We thus rely on locality sensitive
%hashing (LSH)~\cite{DBLP:conf/compgeom/DatarIIM04} to address this issue.
%Specifically, we hash each signature and group together the colliding ones.
%%those that % collide.  These are more likely to be similar; we only need to
%perform comparisons among the nodes in each cluster, ignoring cross cluster
%comparisons.  This approach avoids $O(n^2)$ operations and appears to perform
%well in practice.

%\subsection{Data Discovery Components}
%
%The data discovery module consists of two conceptually different components, a
%data profiler and a graph builder, that collaborate to build the linkage graph
%indicated in Figure~\ref{fig:arch}. 
%
%\begin{enumerate}
%\item {\bf Data profiler.} This component summarizes each column of each table
%into a profile. Each profile contains a signature, which is a domain-dependent,
%compact representation of the original content.  For numerical data, we use the
%data distribution, and for textual data,  a vector of the most significant words
%in the column.  Our profiles also contain information about data cardinality,
%data type, and numerical ranges if applicable~\cite{profiling_survey}.
%
%\item {\bf Graph builder.} Using the profiles, the graph builder finds
%relationships among nodes and represents them as edges. Examples of
%relationships are \emph{content similarity} based on measuring the similarity
%distance between signatures and \emph{schema similarity} that captures the
%similarity of the labels of the different nodes. Another kind of relationship
%captures the hierarchical relationship between a table and its columns.
%\nan{Each node is a column, or a table? The graph is not formally defined.}
%\sibo{The definition of node appears in Section \ref{introduction}, but it will
%be better to recap the definition again.}
%
%\end{enumerate}
%
%The graph resulting from this process can be used to answer discovery queries.
%Discovery queries vary from simple keyword search, useful to quickly glance over
%data, to more involved queries such as table augmentation, to add attributes of
%interest to known tables. We discuss the queries at the end of the section. 

%\subsection{Data Profiling at Scale}

To run at scale, data profiling relies on sampling to estimate the cardinality of
columns and the quantiles for numerical data. This avoids having to sort each
column, which is not a linear computation.  This also allows us to avoid making
a copy of the data,  reducing memory pressure.

The profiler consists of a pipeline of stages. The first stage
performs basic de-noising of the data, such as removing empty values and dealing
with potential formatting issues of the underlying files, \eg figuring out the
separator for a CSV file.  The second stage determines the data type of each
column, information that is propagated along with the data to the next stages.
The rest of the stages are responsible for computing cardinalities, and performing
quantile estimation, \eg to determine numerical ranges.

The profiler has connectors to read data from different data sources, such as
HDFS files, RDBMS or CSV files. Data sets are streamed through the profiler
pipeline;  the pipeline outputs the results
%and  at the end the resulting attribute profile is sent
to a {\it profile store} where it is later consumed by the graph builder, which
will be explained in Section \ref{subsec:graphbuild}.
%At the same time,
The profiler works in a distributed environment using a lightweight coordinator
that splits the work among multiple processors, helping to scale the
computation.

%The discovery module benefits from our distributed architecture that allows us
%to parallelize profiling as well as graph building, further speeding up
%execution times. Other kinds of relationships such as finding FK-PKs are too
%expensive to compute at scale for the whole graph.  These are left to the
%stitching module, which is described next.


\subsection{Graph Builder}
\label{subsec:graphbuild}

The linkage graph is a hyper-graph where each simple node represents a column, each hyper-node consists of multiple simple nodes and can represents a table or a compound key, and each simple edge represents a relationship between two nodes in the graph. Note the hyper-graph can express relationships involving more than two columns. This is useful to capture relationships amongst tables, \eg \pkfk relationship between compound keys.

Examples of relationships are 
\emph{column similarity}, 
\emph{schema similarity}, 
\emph{structure similarity}, % based on SimRank~\cite{DBLP:conf/kdd/JehW02}\mourad{No need to mention how since we don't for the others.}, 
\emph{inclusion dependency}, 
\emph{\pkfk relationship}, and
\emph{table subsumption}. 
The node label contains table metadata, such as table name, cardinality and
other information computed by the profiler. The edge label includes metadata
about the relationship it represents, \eg type and score, if any. Computing
the different relationships requires different time complexities. We 
categorize them into, \textit{light relationships}, which can be computed in
sub-quadratic time, and \textit{heavy relationships}, which needs at least
quadratic time. The light relationships contain column similarity and schema
similarity. The heavy relationships include the \pkfk (PK-FK), inclusion dependency, and structure similarity. 


The graph builder first computes the light relationships amongst pairs of
nodes in advance during offline stage. In particular, the graph builder first finds column similarity and
schema similarity relationships.  Each edge is assigned a weight that represents
the strength of the relationship: the particular meaning depends on the edge
semantics. Also, edges with a similarity less than a user-defined threshold are
discarded to avoid having a graph with too many edges that are not significant.
A straightforward computation of such relationships requires a $O(n^2)$
computation, which will not scale. However, for light-relationships, which depend on similarity metric
such as edit distance and Jaccard similarlity, we can employ locality sensitive hashing
(LSH)~\cite{DBLP:conf/compgeom/DatarIIM04} to yield sub-quadratic runtime; we have found LSH to run quite well in practice, even for data sets up to several Terabytes.

It is usually expensive to compute the heavy relationships. To address this
issue, the graph builder adopts the following general approach: (i)~the
computation of heavy relationships is performed in the background, (ii)~data
profiles and light relationships are used as much as possible to prune the
search space for the heavy relationships, and (iii)~after the user chooses the data of
interest, if the heavy relationships are not ready amongst this data, the graph builder can construct the heavy relationships online.

%For efficiently computing the heavy relationships, 
We use the \pkfk relationships as an example of how
we efficiently compute a heavy relationship and deal with the complexities that arise in real data where errors and noise make finding
such relationships tricky. The graph builder utilizes an
implementation of inclusion dependency (i.e., column A {\it includes all the values of} column B) to find candidate PK-FK relationships, since a PK should include all values in an FK column.  We then 
 refine the candidates using machine learning methods. However,
as it is well known, data in the wild is quite dirty, which makes discovering
inclusion dependencies difficult. Specifically, foreign keys may not match a
primary key, because of errors in either the PK or the FK. To tolerate errors,
we extend traditional inclusion dependency discovery by both key coverage and
text similarity and propose an \emph{\eind} approach. 
We describe this approach in Section~\ref{subsec:eind} and the machine learning
techniques to refine the candidate PK-FK relationships in Section~\ref{subsec:refine}.

\subsection{Error-Robust Inclusion Dependency}\label{subsec:eind}

As noted above, \Pkfk relationships are usually identified using inclusion
dependency techniques. To overcome the presence of dirty data, we propose an
error-robust inclusion dependency scheme.
%, which enhances traditional inclusion dependency techniques.
Consider two columns (or compound columns) from two tables \R and \S denoted by
\RX and \SY. If there is a foreign key constraint on \RX with reference \SY, all
the values in \RX must appear in \SY, which yields an inclusion dependency from
\RX to \SY. However, in the real world, values in foreign key fields may not
exist in the primary key fields due to errors.  In this case an inclusion
dependency from foreign key to primary key does no hold.

To address this, for each distinct value in \RX, we calculate the
text similarity to values in \SY and use the maximum value as the strength of a
value matching. We then compute the total strength of the maximum matching value
divided by the number of values in \RX. This is the overall strength of the
inclusion dependency.  If this number exceeds a predefined threshold $\delta$,
we add an inclusion dependency from \RX to \SY in our linkage graph.
%\nan{This is very weak. We can also tolerate errors by saying if $> x\%$ \RX
%appears in \SY, then there is a related inclusion dependency.}

When dealing with compound columns, we can utilize different text similarity functions on different fields. 
Also, we must compose the individual column scores to achieve an overall strength.


\subsection{Candidate PK-FK Relationships Refinement}
\label{subsec:refine}


The error-robust inclusion dependency algorithm returns a collection of candidate PK-FK relationships.
We use the algorithms in the work by Rostin et al.~\cite{DBLP:conf/webdb/RostinABNL09} 
%\nan{Ahmed's comment: Which is this system? Can it be named?}\mourad{It doesn't have a name!} 
to refine our candidate selection. The authors proposed 10~different features to distinguish foreign key constraints from spurious inclusion dependencies. Consider two columns \RX and \SY with an inclusion dependency from the first one to the second one. 
%In~\cite{DBLP:conf/webdb/RostinABNL09}, 
The specified features include 
\emph{coverage}, the ratio of distinct values in \RX that are contained in \SY, 
\emph{column name similarity}, the similarity between the attribute names of X and Y, 
and \emph{out-of-domain range}, the percentage of values in \SY not within $[\min(\RX), \max(\RX)]$ where $\min(\RX)$ and $\max(\RX)$ are respectively the minimum and maximum values in \RX.

Using the above features, we implemented the four machine learning classification algorithms from Rostin et al.~\cite{DBLP:conf/webdb/RostinABNL09}. These allow us to distinguish spurious inclusion dependencies from real ones with high accuracy. 

Next we discuss how to utilize the linkage graph to help the user discover his interesting data in Section~\ref{sec:discovery}. The PK-FK relationships can be used to link the tables of user interest from the discovery module. However, there may exist multiple subgraphs in the linkage graph that can connect all the interesting tables, which we call \emph{join paths}. In Section~\ref{sec:curating}, we discuss the choice of which one(s) to use to materialize a view for the user.

% !TEX root = ../main.tex
\section{Polystore Query Processing}
\label{sec:curating}

\dcv uses the BigDAWG polystore~\cite{DBLP:journals/pvldb/ElmoreDSBCGHHKK15}. 
BigDAWG  consists of a middleware query optimizer and executor, and shims to various local storage systems.
%, as noted in~\cite{DBLP:journals/sigmod/DugganESBHKMMMZ15,DBLP:journals/pvldb/ElmoreDSBCGHHKK15}. 
We assume that a user has run discovery and that the corresponding linkage graph has been computed. 
We can thus identify a collection of join paths that can materialize a composite table of interest to the user.  
Also, assume that the user has identified the subset of each source table in which he is interested. 
For any join path, it is now straightforward to construct the BigDAWG query that materializes the view specified by each join path. 

\subsection{Selecting a View to Materialize}

\nan{Our solution for selecting a view to materialize is not well justified.}

The conventional data federation wisdom is to choose the join path that minimizes the query processing cost. However, this ignores data cleaning issues, to which we now turn.
To achieve high quality results, one has to clean the data prior to querying the table. 
For example, if one has a data value, \code{New Yark}, and wants to transform it to one of its airport codes \code{(JFK, LGA)}, then one must correct the data to \code{New York}, prior to the airport code lookup. Obviously, cleaning usually entails a human specification of the corrected value or a review of the value produced by an automatic algorithm. Hence, it is expensive in human resources, which we believe is generally ``the high pole in the tent''. 
%\sibo{I made a trial to change to the view selection strategy following the suggestions in the email:}
%However, it is not always preferable to return only the highest quality answer. For example, a user may be willing to sacrifice the answer accuracy to trade more returned (discovered) data.  As such, we provide a multi-faceted algorithm which returns multiple join paths and let the user choose which one to explore. 
As such, the goal of \dcv is to choose the join path that produces the highest quality answer and not the one that is easiest to compute. \srm{Don’t I care about completenetss of match as well?}


In \dcv, a user has to decide how to trade off data quality and cleaning cost. 
\dcv defines two parameters under the user control.

\begin{enumerate}
\item Minimize cost for a specific cleanliness metric. In this case, the user requires the data to be a certain percentage, \emph{P}, correct and will spend whatever it takes to get to that point.

\item Maximize accuracy for a specific cost. In this case, the user is willing to spend \emph{M} and wishes to make the data as clean as possible.
\end{enumerate}

%\sibo{More facets probably come here.}

Sometimes the user is the one actually cleaning the data. In this case, (s)he can use \emph{P} and \emph{M} to quantify the value of her/his time. 
In other cases, cleaning is performed by other domain experts, who generally need to be paid. In this case, \emph{P} and \emph{M} are statements about budget priorities.



As a result, \dcv must make the following decisions.  First, it must make an assessment of the cleanliness of the result of any given join path.  We treat this issue in Section~\ref{subsec:model}. The obvious conclusion is to choose the join path that produces the highest quality result.
Then, we need to choose where to place,  in the resulting query plan, data cleaning operations to be the most efficient.  This is the topic of Section~\ref{subsec:gain}.


\subsection{The Cleanliness Model}
\label{subsec:model}

In \dcv, we collect information about the errors in each data set.\mourad{Do these assumptions makes sense!? I doubt anyone at Merck will volunteer to do it or even knows how to do it!} 
First, we assume that each data set owner gives us accuracy metrics for each column, namely an estimation for the percentage of the column which is erroneous. 
Second, the data owner also specifies a ``disorder metric" that indicates the average and variance of the ``lexical distance'' between an incorrect value and its ground truth. 
For example, if salary errors average 5\% with standard deviation 2\%, then $\frac{2}{3}$\sibo{Do not get how the 2/3 comes. The distribution is not given in the context}  of the errors are less than 7\%. 
If an address field routinely confuses ``\textit{road}'' and ``\textit{street}'', but very rarely gets the name of the street wrong, then the lexical distance is again very small. \srm{We should explicitly define different metrics for lexical and numeric.  Mean +/- std dev doesn’t make sense for lexical, I think.}

To answer a query, \dcv generates a sub-graph for the query from the available join graph (which is the sub-graph of the linkage graph that only contains FK-PK edges) that covers all the joins in the query. 
For a given query, there may be multiple sub-graphs. 
\dcv chooses the one with the maximum cleanliness, according to the following cleanliness model.


We first define the cleanliness of an edge in the join graph. 
Given two linked nodes \RX and \SY in the graph, when we align the values in \RX and \SY using the maximum matching  defined in Section~\ref{subsec:eind}, we can tolerate some errors. Suppose that the lexical distances of the two nodes are \Dis(\RX) and \Dis(\SY), respectively\srm{is this an average over all values?}. The distance between two values in \RX and \SY that represent the same object is highly likely\srm{Not really...} to be within $\Dis(\RX) + \Dis(\SY)$. Thus we limit the weight (i.e., the text similarity) of each alignment in the maximal matching to be no smaller than $1-(\Dis(\RX)+\Dis(\SY))$\srm{This could be < 0.  This is for lexical similarity;  what about for numeric quantities.}. 
Then the fraction of the values in the foreign key that appear in the maximal matching is the cleanliness of the edge. We define the cleanliness of a node \RX simply as its accuracy \Acc(\RX), which means the percentage of values in \RX that are not erroneous.\srm{No, DIS(R[X]) is the average dissimilaritity of the edge;  it doesn’t say anything about whetehr a given value is erroneous.}

Given a join path, its cleanliness is obtained by multiplying the cleanliness values of all the edges and nodes in the join path. Considering the join path is necessary since errors can propagate along this path and is dominated by the dirtiest one.



\subsection{Query Plan with Data Cleaning Operation}
\label{subsec:gain}

Obviously expensive cleaning should be performed on as few records as possible. Hence, we would like to insert cleaning operations as late in the query plan as possible. 
Unfortunately, if we run a query with the following predicate:
\vspace{.5em}
\dots \textsf{where} $name = $ ``\code{New York}''
\vspace{.5em}
\noindent Then the misspelled city name,  \ie \code{New Yark}, will not be found, and accuracy will suffer. One solution is to clean the entire source data sets to avoid such errors, an expensive proposition indeed. If the penalty is small, then we can insert cleaning ``late'' in the query plan\srm{Why not use a relaxed similarity threshold and then clean those nodes.}. On the other hand, if the penalty is large, then we must insert cleaning earlier, even though at much higher cost. We formalize our idea as follows.


For each edge in the join path, if we put the cleaning operation ahead of the join operation, i.e., we only clean the tuple pairs in our maximal matching, we spend cleaning cost linear to the number of foreign keys and can only expect to achieve the same accuracy  as the cleanliness of the edge. In contrast, if we first clean all the possible tuple pairs of the foreign key and primary key and then generate the real join results, we can achieve 100\% accuracy with a cleaning cost quadratic to the number of foreign keys. Similarly, if we put the cleaning operation behind the where clause, we can only expect to achieve the same accuracy of the column with the cost proportional to the number of tuples satisfying the where clause. Otherwise, we can achieve 100\% accuracy with the cost proportional to the number tuples in the whole column.


In this setting, we can develop a dynamic programming algorithm to obtain a query plan with cleaning operations that achieve the maximum accuracy gain with limited cleaning cost budget or achieve the desired accuracy with the smallest cleaning cost budget\srm{What is algorithm?  Really dynamic programming?}. In the algorithm, there are three kinds of states for each edge and each node in the join graph: 
no cleaning, cleaning before query operation, and cleaning after query operation, each with different cleaning costs.
%do not apply cleaning operation on it, apply cleaning operation before the query operation, and apply cleaning operation after the query operation, each with different cleaning costs.



\section{Cleanliness Estimation}
\label{sec:cleanliness}

% !TEX root = ../main.tex
\section{Enhanced Data Stitching}
\label{sec:enhancedstitching}

% !TEX root = ../main.tex
\section{Updates}
\label{sec:updates}

Real-world data is rarely static, which is exactly the scenario that \dcv faces at Merck and the MIT Data Warehouse.  We consider three types of updates.

\stitle{(1) Insertions/deletions on source tables.} This happens when there is a change to a table, \eg insertion of a new procurement record in the MIT Data Warehouse. This may also happen when data sources are cleaned (see Section~\ref{sec:curating}).


\stitle{(2) Replacement of source tables.} Large companies typically rely on both internal and external information to build their knowledge bases. For instance, Merck collects published standard medical names from the World Health Organization (WHO) to help construct their own ontology. These data sources are updated periodically by WHO.  Sometimes, even the format may be changed, e.g., from a JSON file to a CSV file.


\stitle{(3) Updating MVs.} MVs might be created based on other MVs.  Since cleaning effort can happen at any place in a query plan, MVs may need to be updated.


%In response to the above three types of updates, 
\dcv uses three corresponding strategies to cater for the above updates.


\stitle{(i) MV maintenance.} \dcv will leverage mature techniques for maintaining materialized views (see~\cite{DBLP:journals/debu/GuptaM95} for a survey).  In this way, \dcv incrementally propagates updates through the data curation pipeline to update downstream MVs.




\stitle{(ii) Provenance management.}  In case (2) above, the human effort to update and clean MVs may be daunting. In this case, MVs should be discarded rather than updated. Naturally, there is a need for a workflow component that supports data versioning and branching operations. \dcv will leverage Decibel~\cite{DBLP:journals/pvldb/MaddoxGEMPD16}, a system developed at MIT for this purpose.

\stitle{(iii) Descriptive and prescriptive data cleaning.} When a scientist updates an MV (case 3) above), this will trigger the update propagation to descendant MVs, as well as back upstream to data sources, if possible. To perform this propagation, we plan to leverage the techniques in DBRx~\cite{DBLP:conf/sigmod/ChalamallaIOP14}, a system developed by QCRI and Waterloo.


%As data appears incrementally, we need an incremental inclusion dependency algorithm. Extending our algorithms in Section~\ref{subsec:eind} is a project we are currently working on.


% !TEX root = ../main.tex
\section{Tractable Curation Workflow}
\label{sec:workflow}

The process of data curation may entail multiple iterations of discovery,
stitching, querying, and curation. Cleaning and
transformation procedures must be guided by the user. \dcv will use a fairly
conventional workflow system that will allow a human to construct sequences of
operations, undo ones that are unproductive, and utilize branching to try
multiple processing operations.


In addition, we plan to build a workflow orchestrator which will retain previous
operation sequences and propose ones that best fit a new situation. We
expect this tactic to be successful because the types of data curation and
preparation steps in an enterprise often follow repetitive patterns.
Specifically, our workflow orchestrator will store the user query, the sequence
of operations in a central workflow registry together with the meta-data and
signatures provided by the data discovery component. 


The orchestrator will evaluate a new user query and the initial results of
the data discovery component against the workflow registry. Similar queries and
similar data profiles are likely to demand similar cleaning procedures. Previous
workflows will be ranked based on the similarity of the user query and retrieved
discovery results. Accordingly, the related curation and preparation procedures
will be shown in ranked order to the user.

% !TEX root = ../main.tex
\section{Data Civilizer in the Wild}
\label{sec:wild}

We have built an initial prototype of \dcv containing the data discovery module
and the linkage graph builder as discussed in Section~\ref{sec:discovery} and~\ref{sec:stitching}. Ultimately, we will integrate \dcv with
\texttt{BigDawg}, our implementation of a polystore, to realize the end-to-end
system described in this paper.  We have been working closely with end users to
adapt our prototype to relevant real world problems.  In particular, we have deployed
a preliminary prototype of discovery in two different organizations The MIT Data
warehouse (MIT DWH) is a group at MIT responsible for building and maintaining a
data warehouse that integrates data from multiple source systems. Merck, a big
pharmaceutical company, manages large volumes of data, which are handled by
different storage systems.  In the following, we first outline the current state
of the system and the modules that are currently deployed.  We then provide
details on each organizations' requirements, and how we are using \dcv to help
them.

\subsection{Current \titledcv Prototype} 

The current prototype runs as a
server and can access data that resides in a file system or one or more
databases.  Clients  connect to \dcv through a RESTful API.  The current
deployment of \dcv  includes both  the graph builder with the error-robust
inclusion dependency discovery from Section~\ref{sec:stitching} and the
discovery component described in Section~\ref{sec:discovery}.  Thus a user is
able to submit queries that consist of attribute names, attribute values, or
tables, and receive different types of results based on the selected
similarity function. The similarity function depends on the specific use case,
which we will discuss based on our two use cases MIT and Merck.  We plan to
demo our \dcv prototype at the conference.

\subsection{MIT Data Warehouse}

One of the key tasks of the MIT DWH team is to assist its customers --
generally MIT administrators -- to answer any of their questions about MIT
data. For example, staff usually want to create reports, for which they need
access to various kinds of data. The warehouse contains around 1TB of data
spread across approximately 3K tables.

In their current workflow, a
DWH customer presents a question, which a DWH team
members by manually searching for tables containing relevant data. Once they have determined
the tables of interest, they create a view that is accessed by the customer to
solve the question at hand. Below are some of the common use cases we have
encountered:

\mypar{Fill in virtual schema} When a customer arrives with a question such as:
\emph{I need to create a report with the \textbf{gender distribution of the
faculty per department and year}}, the data warehouse personnel can use \dcv to
find all the tables that contain schema names similar to the attributes exposed
by the query, \eg gender, faculty name, department, and year.

\mypar{Table redundancy} Multiple views are created for different customers.
Many of them contain very similar data, as multiple customers are
interested in similar items. To reduce the redundancy of data, \dcv helps
detect complementary as well as repeated sources. This sheds light on the status
of the warehouse and helps to maintain it tidy and minimal.

Our prototype is deployed in an Amazon EC2 instance managed by the MIT DWH team
with access to their data. We have been working with them for 3 months,
iterating over priorities and learning about the problems they are facing.

In the future, we aim to deploy the entire \dcv system to enable running queries
directly over the hundreds of data sources, by using the polystore query
processing and curation module and linkage graph builder to create the necessary
views on demand. 

\subsection{Merck}

Merck is a large pharmaceutical company that manages massive volumes of data spread
across around 4K databases in addition to several data lakes. 
%The use cases are varied, however there is a common case. 
One of the data assets of any pharmaceutical company are internal databases of
chemical compounds and structures. Usually, these are more valuable when
integrated with external, well-known and curated databases, such as
PubChem~\cite{pubchem}, ChEMBL~\cite{ChEMBL}, or DrugBank~\cite{DrugBank}. We
describe two common use cases that occur in this context:

\mypar{Enrich data} One of the reasons for the existence of multiple chemical
databases is that each puts an emphasis on different information. Analysts
typically face situations in which they are interested in a set of attributes
that are spread across different tables on different databases. \dcv helps to
detect such attributes and bring them together \emph{on-demand}  to serve the
users' purpose.

\mypar{Identify entities} One single chemical entity may be referred to with
different identifiers and formats in different databases. Chemical identifiers
have been a subject of research in the bioinformatics community: multiple
different formats have been proposed with different properties according to the
scenario. \dcv helps them on this task by discovering datasets that contain
schemas with the entities of interest.

We have been collaborating with 4 engineers and bioinformatics experts at Merck
during the last 2 months. During this period we have learned about their use
cases and we have used discovery on chemical databases that are publicly
available. In the future, we plan to integrate internal datasets too, with the
goal of evaluating \dcv with more varied datasets.




\bibliographystyle{abbrv}

\balance\footnotesize
\bibliography{main}
\vskip 1em

\end{document}

% End

%%% Local Variables:
%%% mode: latex
%%% TeX-master: t
%%% End:
